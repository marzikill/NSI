\documentclass{article}

\usepackage[latin1]{inputenc}    
\usepackage[T1]{fontenc}

\usepackage[bloc,completemulti]{automultiplechoice}    
\usepackage{multicol}
\begin{document}

\AMCrandomseed{1237893}

\element{amc}{
  \begin{question}{licence}\bareme{b=2}
    Sous quelle licence AMC est-il distribu� ?
    \begin{multicols}{2}
      \begin{reponses}
        \bonne{GNU General Public License V3}
        \mauvaise{GNU General Public License V2}
        \mauvaise{Licence commerciale AMC}
        \mauvaise{Licence Apache}
      \end{reponses}
    \end{multicols}
  \end{question}
}

\element{amc}{
  \begin{questionmult}{logiciels}\bareme{haut=3}
    Sur quels logiciels repose l'impl�mentation d'AMC version 0.199 ?
    \begin{multicols}{2}
      \begin{reponses}
        \bonne{\LaTeX}
        \bonne{Perl}
        \bonne{ImageMagick ou GraphicsMagick}
        \mauvaise{Apache}
        \mauvaise{Firefox}
        \mauvaise{Gimp}
      \end{reponses}
    \end{multicols}
  \end{questionmult}
}

\element{amc}{
  \begin{questionmult}{taches}\bareme{haut=3}
    Quelles sont les t�ches qui peuvent �tres effectu�es de mani�re
    automatique par AMC ?
    \begin{multicols}{2}
      \begin{reponses}
        \bonne{La saisie � partir des scans des copies}
        \bonne{La notation des copies}
        \bonne{La production d'une correction individualis�e des copies}
        \bonne{La production d'un fichier de notes au format OpenOffice}
        \mauvaise{La pr�paration d'une tasse de caf� pour l'enseignant}
      \end{reponses}
    \end{multicols}
  \end{questionmult}
}

\element{amc}{
  \begin{question}{paquet}\bareme{b=2}
    Un paquet logiciel est fourni sur le site d'AMC. Quel en est le format ?
    \begin{reponseshoriz}
      \bonne{deb}
      \mauvaise{rpm}
      \mauvaise{exe}
      \mauvaise{slp}
    \end{reponseshoriz}
  \end{question}
}

\exemplaire{10}{    

%%% debut de l'en-t�te des copies :    

\noindent{\bf Classe d'application d'AMC  \hfill Examen du 01/01/2010}

\vspace{2ex}


  Cet examen a pour but d'illustrer l'utilisation d'\emph{Auto
    Multiple Choice}. Vous pourrez trouver sur le site d'AMC les
  copies de Jojo Boulix et Andr� Roullot afin de tester la saisie
  automatique, ainsi que le fichier listant les �tudiants de la classe
  d'application d'AMC (dont font partie Jojo et Andr�) afin de tester
  l'association automatique � partir des num�ros d'�tudiants.

  Si vous choisissez une note maximale de 10 et l'arrondi normal pour
  cet examen, Jojo obtiendra la note 5/10 et Andr� la note
  6/10.


\vspace{3ex}

\noindent\AMCcode{etu}{8}\hspace*{\fill}
\begin{minipage}{.5\linewidth}
$\longleftarrow{}$ codez votre num�ro d'�tudiant ci-contre, et �crivez votre nom et pr�nom ci-dessous.

\vspace{3ex}

\champnom{\fbox{    
    \begin{minipage}{.9\linewidth}
      Nom et pr�nom :
      
      \vspace*{.5cm}\dotfill
      \vspace*{1mm}
    \end{minipage}
  }}\end{minipage}

\vspace{1ex}

\noindent\hrulefill

\vspace{2ex}

\begin{center}
  Les questions faisant appara�tre le symbole \multiSymbole{} peuvent
  pr�senter z�ro, une ou plusieurs bonnes r�ponses. Les autres ont une
  unique bonne r�ponse.
\end{center}

\noindent\hrulefill

\vspace{2ex}
%%% fin de l'en-t�te

\melangegroupe{amc}
\restituegroupe{amc}

\clearpage    

}   

\end{document}
