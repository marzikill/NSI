%! TEX program = xelatex
% Nécessaire pour la compilation pstricks
\documentclass[10pt]{article}                   % autres choix : report, book
\usepackage[a4paper]{geometry}                  % taille correcte du papier
\usepackage{silence}                            % Filtrer tous ces warnings ennuyeux
\usepackage[utf8]{inputenc}                     % encodage du fichier source
\usepackage[T1]{fontenc}                        % gestion des accents (pour les pdf)
\usepackage[french]{babel}                    % rajouter éventuellement english, greek, etc.
\usepackage{eurosym}                            % symbole euro

\usepackage{textcomp}                           % caractères additionnels
\usepackage{lmodern}                            % remplacer éventuellement par txfonts, fourier, etc.
\usepackage{amsmath,amssymb, amsthm}            % tout ce qu'il faut pour écrire des maths
\usepackage{dsfont}                             % Pour la fonction indactrice
\usepackage{xfrac}                              % fraction d'anneau

\usepackage{adforn}                             % Ornements header
\usepackage{yfonts}                             % Fontes gothiques supplémentaires
\usepackage{fourier-orns}                       % Ornements header
\usepackage{mathpazo}
\usepackage{pstricks,pst-plot,pstricks-add}     % tracer des fonctions
\usepackage[linesnumbered, french]{algorithm2e} % mise en page des algorithmes
\usepackage{tikz, tkz-euclide, tkz-tab}         % Dessiner aver tikz
\usetkzobj{all}                                 % et tout plein de paquets utiles
\usetikzlibrary{arrows.meta}
\usetikzlibrary{automata,positioning}
\usetikzlibrary{decorations.pathreplacing,calc}
\usetikzlibrary{arrows}
\usetikzlibrary{calc}
\usetikzlibrary{decorations.fractals}           % Dessiner les fractales usuelles
\usepackage{enumitem}                           % 1, 2, 3...
\usepackage{pgffor}                             % Pour faire des boucles

\usepackage{multicol}
\usepackage{comment}
\usepackage{array}                              % tableaux un peu plus paramétrables
%\usepackage[makestderr]{pythontex}             % lors d'un qcm, désactiver pythontex
\usepackage[bloc, completemulti, nowatermark]{automultiplechoice}
\usepackage{minted}

\usepackage{graphicx}                           % pour inclure des images
\usepackage{xspace}                             % protection des espaces
\usepackage{xcolor}                             % pour gérer les couleurs
\usepackage{microtype}                          % améliorations typographiques
\usepackage{hyperref}                           % gestion des hyperliens
\hypersetup{
    colorlinks = true,
    linkcolor = blue!50!gray!100!,
    linkbordercolor = {white},
}
\hypersetup{pdfstartview=XYZ}                   % Zoom hyperref
 
%%%%%%%%%%%
%% Maths %%
%%%%%%%%%%%
% Raccourcis pour les trucs longs à écrire. Reste d'Oxford principalement. 
\newcommand{\Prob}[2][]{\mathbb{P}_{#1}\left( #2 \right)}
\newcommand{\Exp}[2][]{\mathbb{E}_{#1}\left[ #2 \right]}
\newcommand{\Lawconv}{\stackrel{\mathcal{L}}{\longrightarrow}}
\newcommand{\Var}[1]{\mbox{Var}\left(#1 \right)}
\newcommand{\Card}[1]{\mbox{Card}\left(#1 \right)}
\newcommand{\Int}[4]{\int_{#1}^{#2} \! {#3} \, \mathrm{d}{#4}}
\newcommand{\kw}{\;|\;}
\newcommand{\im}{\textrm{Im}}
\newcommand{\disclamer}{\textit{L’usage de toute calculatrice, téléphone portable, ordinateur est strictement interdit. Un soin tout particulier devra être porté à la qualité et à la précision de la rédaction des argument } \vspace{1.cm}}
\newcommand{\legendeexos}{
\paragraph{Légende des exercices.}
\begin{center}
\begin{tabular}{cccc}
$\clubsuit$ : très flacile & $\varheart$ : à savoir par cœur & $\vardiamond$ : care aux calculs & $\spadesuit$ : qui s'y frotte s'y pique 
\end{tabular}
\end{center}
}

%%%%%%%%%%%%%%%%
%% Structures %%
%%%%%%%%%%%%%%%%

\newcounter{dfcount}
{\theoremstyle{definition} \newtheorem{definition}[dfcount]{Définition}}
\newcounter{propcount}
\newtheorem{propriete}[propcount]{Propriété}
\newcounter{thcount}
\newtheorem{theoreme}[thcount]{Théorème}
\newcounter{excount}
{\theoremstyle{definition} \newtheorem{exemple}[excount]{Exemple}}
\renewcommand{\theexemple}{\arabic{exemple}}
% Environnement exos
\newcounter{exoscount}
\newtheoremstyle{exercicestyle} % name of the style to be used
  {20pt}                        % measure of space to leave above the theorem. E.g.: 3pt
  {3pt}                         % measure of space to leave below the theorem. E.g.: 3pt
  {}                            % name of font to use in the body of the theorem
  {}                            % measure of space to indent
  {\bfseries}                   % name of head font
  {.}                           % punctuation between head and body
  {.5em}                        % space after theorem head; " " = normal interword space
  {}                            % Manually specify head
\theoremstyle{exercicestyle}
\newtheorem{exos}[exoscount]{Exercice}
\newenvironment{exo}[2]
  {
   \begin{exos}[#1]
   \leavevmode
   \marginpar{\hfill $#2 $ }}
  {\end{exos}}

\newenvironment{prob}{ \begin{center}
\Large{ \textsc{Problème}} \end{center} }{ \sepdown }

 % Théorèmes/propriétés avec un retour à la ligne.
\newtheoremstyle{break2}
  {\topsep}%   Space before theorem
  {\topsep}%   Space after theorem
  {\itshape}%  Body font
  {}%          Indent amount (empty = no indent, \parindent = para indent)
  {\bfseries}% Thm head font   
  {}%         Punctuation after thm head
  {.5em }%  Space after thm head (\newline = linebreak)
  {\thmname{#1}\thmnumber{ #2}\thmnote{ (#3)}.}% Thm head spec

\theoremstyle{break2}
\newtheorem{fbth}[thcount]{Théorème}
\newenvironment{bth}[1][]{\begin{fbth}[#1]\item\relax}{\end{fbth}}

\newtheorem{fbprop}[propcount]{Propriété}
\newenvironment{bprop}[1][]{\begin{fbprop}[#1]\item\relax}{\end{fbprop}}

 % Théorèmes/propriétés avec un retour à la ligne.
\newtheoremstyle{break3}
  {\topsep}%   Space before theorem
  {\topsep}%   Space after theorem
  {}%  Body font
  {}%          Indent amount (empty = no indent, \parindent = para indent)
  {\bfseries}% Thm head font   
  {}%         Punctuation after thm head
  {.5em }%  Space after thm head (\newline = linebreak)
  {\thmname{#1}\thmnumber{ #2}\thmnote{ (#3)}.}% Thm head spec

\theoremstyle{break3}
\newtheorem{fbdf}[dfcount]{Définition}
\newenvironment{bdef}[1][]{\begin{fbdf}[#1]\item\relax}{\end{fbdf}}

% Répéter un exercice n fois
\newcommand{\looped}[2]{
	\foreach \n in {1,...,#1}{
		#2
	}
}
%%%%%%%%%%%%%%%%%%
%% Mise en page %%
%%%%%%%%%%%%%%%%%%

%\setlength{\textheight}{26.2cm}
%\setlength{\textwidth}{17.0cm}
%\setlength{\oddsidemargin}{-0.5cm}
%\setlength{\evensidemargin}{0.0cm}
%\setlength{\topmargin}{-2.0cm}
%\pagestyle{empty}
%\newif \ifnote
%\notefalse
%\reversemarginpar

\usepackage{titlesec}      % Interlignes pas trop gros
\titlespacing{\paragraph}{ %
  0pt}{                    % left margin
  1\baselineskip}{         % space before (vertical)
  1em}                     % space after (horizontal)

\usepackage{titlesec}
%\titleformat
%   {\section} % <command>
%   {\normalfont\Large\bfseries} % <format>
%   {\thesection} % <label>
%   {1em} % <sep>
%   {} % <before-code>
%\titleformat{\section}{\bfseries\sffamily\large}{\thesection.}{\hspace{1cm}}{}%
\titlespacing*                  % starred version: first paragraph is not indented
    {\section}                  % <command>
    {-22pt}                     % <left>
    {3.5ex plus 1ex minus .2ex} % <before-sep>
    {2.3ex plus .2ex}           % <after-sep>


\setlength{\columnseprule}{1pt}        % Séparateur vertical dans multicol
\def\columnseprulecolor{\color{black}} % De couleur noire.

% Type de colonnes pour array
\newcolumntype{P}[1]{>{\centering\arraybackslash}p{#1}}
\newcolumntype{M}[1]{>{\centering\arraybackslash}m{#1}}

% Filtrer les messages d'erreur kisoulent 
\WarningFilter{latex}{Marginpar on page}
\WarningFilter{latexfont}{Size}

%%%%%%%%%%%%%%%%%%%%%%%%%
%% Faire du joli latex %% 
%%%%%%%%%%%%%%%%%%%%%%%%%

% Faire des cœurs et des carreaux 
\DeclareSymbolFont{extraup}{U}{zavm}{m}{n}
\DeclareMathSymbol{\varheart}{\mathalpha}{extraup}{86}
\DeclareMathSymbol{\vardiamond}{\mathalpha}{extraup}{87}

% Police verbatim + souligné (links)
\usepackage[normalem]{ulem}
\useunder{\uline}{\ulined}{}%
\DeclareUrlCommand{\bulurl}{\def\UrlFont{\ttfamily\ulined}}

% Console Python. Les fichiers sont stockés dans ./pyfiles
\newcommand{\expandpyconc}[1]{\expandafter\reallyexpandpyconc\expandafter{#1}}
\newcommand{\reallyexpandpyconc}[1]{\pyconc{exec(compile(open('#1', 'rb').read(), '#1', 'exec'))}}

\newenvironment{pyconcodeblck}[1]%
 {\newcommand{\snippetfile}{./pyfiles/snippet-#1.py}
  \VerbatimEnvironment
  \begin{VerbatimOut}{\snippetfile}}%
 {\end{VerbatimOut}%
  \expandpyconc{\snippetfile}}


\begin{document}
\AMCrandomseed{1237893}

\element{amc}{
  \begin{questionmult}{contenu liste}\bareme{b=1, m=0.5}
    Un tableau peut contenir :
    \begin{multicols}{2}
      \begin{reponses}
        \bonne{Une valeur}
        \bonne{Plusieurs valeurs}
        \bonne{Aucune valeur}
        \bonne{La valeur 0}
      \end{reponses}
    \end{multicols}
  \end{questionmult}
}

\element{amc}{
    \begin{question}{append}\bareme{b=1, m=0.5}
        On tape dans la console les deux instructions suivantes : 
        \inputminted[frame=single, framesep=2mm, label=Console Python]{python}{./pyfiles/append.py}
       Que renvoie la console ?
        \begin{multicols}{2}
            \begin{reponses}
                \mauvaise{\texttt{[6,5,9,3,7,10,2,15]}}
                \bonne{\texttt{[5,9,3,7,10,2,15,6]}}
                \mauvaise{\texttt{[5,9,3,7,6,10,2,15]}}
                \mauvaise{\texttt{[6]}}
            \end{reponses}
        \end{multicols}
    \end{question}
}

\element{amc}{
    \begin{question}{taille}\bareme{b=1, m=0.5}
        Le tableau \texttt{t = [5, 9, 3, 7, 10, 2, 15]} contient :
        \begin{multicols}{2}
            \begin{reponses}
                \bonne{7 valeurs}
                \mauvaise{6 valeurs}
                \mauvaise{15 valeurs}
                \mauvaise{8 valeurs}
            \end{reponses}
        \end{multicols}
    \end{question}
}

\element{amc}{
    \begin{question}{remplacer}\bareme{b=1, m=0.5}
        Soit le tableau \texttt{t = [5, 9, 3, 7, 10, 2, 15]}. On souhaite remplacer la valeur 7 par 20. En Python, on utilise pour cela l'instruction :
        \begin{multicols}{2}
            \begin{reponses}
                \mauvaise{\texttt{L(7<>20)}}
                \mauvaise{\texttt{L[4]=20}}
                \bonne{\texttt{L[3]=20}}
                \mauvaise{\texttt{L[3]==20}}
            \end{reponses}
        \end{multicols}
    \end{question}
}

\element{amc}{
    \begin{question}{rajouter}\bareme{b=1, m=0.5}
        Soit le tableau \texttt{t = [5, 9, 3, 7, 10, 2, 15]}. On souhaite ajouter 2 à tous les éléments du tableau. Le code Python effectuant cette tâche est :
        \begin{multicols}{2}
            \begin{reponses}
                \mauvaise{\inputminted[frame=single, framesep=2mm, label=Code Source]{python}{./pyfiles/ajout1.py}}
                \mauvaise{\inputminted[frame=single, framesep=2mm, label=Code Source]{python}{./pyfiles/ajout2.py}}
                \mauvaise{\inputminted[frame=single, framesep=2mm, label=Code Source]{python}{./pyfiles/ajout3.py}}
                \bonne{\inputminted[frame=single, framesep=2mm, label=Code Source]{python}{./pyfiles/ajout4.py}}
            \end{reponses}
        \end{multicols}
    \end{question}
}


\element{amc}{
    \begin{question}{len}\bareme{b=1, m=0.5}
        On tape dans la console les deux instructions suivantes : 
        \inputminted[frame=single, framesep=2mm, label=Console Python]{python}{./pyfiles/taille.py}
       Que renvoie la console ?
        \begin{multicols}{2}
            \begin{reponses}
                \mauvaise{6}
                \mauvaise{15}
                \bonne{7}
                \mauvaise{5}
            \end{reponses}
        \end{multicols}
    \end{question}
}

\element{amc}{
    \begin{question}{print}\bareme{b=1, m=0.5}
        On donne le code source suivant :
        \inputminted[frame=single, framesep=2mm, label=Code Source,linenos=true]{python}{./pyfiles/affiche.py}
       Après exécution, qu'est-ce qui est affiché par le programme ?
        \begin{multicols}{2}
            \begin{reponses}
                \mauvaise{\texttt{9 3 7 10 2 15}}
                \mauvaise{\texttt{15 2 10 7 3 9 5}}
                \bonne{\texttt{5 9 3 7 10 2 15 }}
                \mauvaise{\texttt{2 10 7 3 9 5}}
            \end{reponses}
        \end{multicols}
    \end{question}
}

\element{amc}{
    \begin{question}{inserer}\bareme{b=1, m=0.5}
        On tape dans la console les deux instructions suivantes : 
        \inputminted[frame=single, framesep=2mm, label=Console Python]{python}{./pyfiles/insert.py}
       Que renvoie la console ?
        \begin{multicols}{2}
            \begin{reponses}
                \mauvaise{\texttt{ [100, 9, 2,7 ,10 ,1,15]}}
                \bonne{\texttt{[ 15, 1 ,10,100, 7, 3 ,9 ,5] }}
                \mauvaise{\texttt{[5,9, 2,100,7, 10 ,2 ,15 ] }}
                \mauvaise{\texttt{[5,9,100,2,7,10,1,15]}}
            \end{reponses}
        \end{multicols}
    \end{question}
}

\element{amc}{
    \begin{question}{mystere}\bareme{b=1, m=0.5}
        On donne le code source suivant :
        \inputminted[frame=single, framesep=2mm, label=Console Python,linenos=true]{python}{./pyfiles/mystere.py}
        Après exécution, qu'est-ce qui affiché par ce programme ?
        \begin{multicols}{2}
            \begin{reponses}
                \mauvaise{La somme des indices du tableau}
                \mauvaise{La moyenne des indices du tableau}
                \bonne{La moyenne des éléments du tableau}
                \mauvaise{La somme des éléments du tableau}
            \end{reponses}
        \end{multicols}
    \end{question}
}
\element{amc}{
    \begin{question}{mystere2}\bareme{b=1, m=0.5}
        On donne le code source suivant :
        \inputminted[frame=single, framesep=2mm, label=Console Python,linenos=true]{python}{./pyfiles/mystere2.py}
        Après exécution, qu'est-ce qui est affiché par ce programme ?
        \begin{multicols}{2}
            \begin{reponses}
                \bonne{15}
                \mauvaise{7}
                \mauvaise{5}
                \mauvaise{49}
            \end{reponses}
        \end{multicols}
    \end{question}
}


\exemplaire{1}{    

%%% debut de l'en-tête des copies :    

\noindent{\bf NSI - Contrôle de connaissances : les tableaux  \hfill Examen du 01/01/2020}

 \paragraph{Pour les profs :} QCM à correction automatique réalisé à l'aide de \textbf{ACM} : 
 \begin{center}
 \bulurl{https://www.auto-multiple-choice.net/}
 \end{center}
 Utiliser ACM et \texttt{minted} est un peu pénible : à cause de conflits avec \texttt{verbatim} il faut utiliser la commande \texttt{\textbackslash{}inputminted\{python\}\{python\_file.py\}} pour chaque morceau de code...
%\vspace{3ex}

\begin{minipage}{0.5\textwidth}
\noindent\AMCcode{etu}{8}\hspace*{\fill}
\end{minipage}
\begin{minipage}{.5\linewidth}
$\longleftarrow{}$ codez votre numéro d'élève ci-contre, et écrivez votre nom et prénom ci-dessous.

\vspace{3ex}

\champnom{\fbox{    
    \begin{minipage}{.9\linewidth}
      Nom et prénom :
      
      \vspace*{.5cm}\dotfill
      \vspace*{1mm}
    \end{minipage}
  }}\end{minipage}

\vspace{1ex}

\noindent\hrulefill
\begin{center}
  Les questions faisant apparaître le symbole \multiSymbole{} peuvent
  présenter zéro, une ou plusieurs bonnes réponses. Les autres ont une
  unique bonne réponse. Un point sera accordé par réponse correcte, un demi point enlevé par réponse incorrecte.
\end{center}
\noindent\hrulefill

\vspace{2ex}
%%% fin de l'en-tête

%\melangegroupe{amc}
\restituegroupe{amc}

}


\end{document}

