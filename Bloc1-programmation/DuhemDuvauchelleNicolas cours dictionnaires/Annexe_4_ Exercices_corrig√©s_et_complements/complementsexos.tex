\documentclass[a4paper,12pt,dvips]{article}
\usepackage[np]{numprint}

\usepackage[T1]{fontenc}
\usepackage[utf8]{inputenc}
\usepackage{fourier}
\usepackage{array,multicol,multirow,enumerate,eurosym,colortbl,fourier}
\usepackage{amsmath}
\usepackage{graphicx,pst-all,color,tabularx}
%\usepackage{lscape}
\usepackage{tabularx}
\usepackage{amsmath,amsfonts,amsthm,amssymb,geometry,variations,tabularx}
%\usepackage{frcursive}
\usepackage{fancyhdr}
\usepackage{lscape}
%\input tabvar.tex
\setlength{\parindent}{0mm}
\def\d{$\diamond \,$}
\def\*{\times}
\def\Euro{\textgreek{\euro}\ }
\def\m@th{\mathsurround=0pt}
\def\trou#1{
    \setbox0=\hbox{\textbf{#1}} \dp0=0pt \m@th
    \underline{\hbox{\hskip\wd0}}n
}
\newenvironment{listerd}%
	{\begin{list}{\Listerd}{}}%
	{\end{list}}
	\newcommand{\oijk}[1][O]{\DecalV[.8pt]{#1;\vec{\imath},\vec{\jmath},\vec{k}\,}\xspace}
\newcommand{\Q}{\mbox{\ding{114}\hspace{-.7em}\raisebox{.2ex}[1ex]{\ding{51}}}}

\newcommand{\VF}{%
	\hfill\makebox[1cm][s]{\bf V\hfill F}
	}
	
\newcommand{\Rep}[1][]{%
	\ifthenelse{\equal{#1}{v}}{%
		\hfill\makebox[1cm][s]{\Q\hfill \ding{114}}}{%
		\ifthenelse{\equal{#1}{f}}{%
			\hfill\makebox[1cm][s]{\ding{114} \hfill\Q}}{%
			\hfill\makebox[1cm][s]{\ding{114} \hfill\ding{114}}}}}

\newcommand{\MaGrille}[5][10]{%
	\psgrid[gridlabels=0,%
	        subgriddiv=0,%
			gridwidth=0.5pt,%
			griddots=#1](0,0)(#2,#3)(#4,#5)}

\newcommand{\MesAxes}[5][.6pt]{%
	{\footnotesize
	\psaxes[labelsep=3pt,ticksize=2pt,linewidth=#1]{->}(0,0)(#2,#3)(#4,#5)
	\pnode(#4,0){XX}\pnode(0,#5){YY}}}

\newcommand{\machine}[4]{
        \begin{pspicture}
        \rput(0,0){\rnode{A}{#1}}
        \rput(3,0.11){\rnode{B}{#2}}
        \psset{nodesep=5pt}
        \ncarc[arcangleA=25,arcangleB=25]{->}{A}{B}\mput*{\ovalnode{m}{#3}}
        \ncarc[arcangleA=25,arcangleB=25]{->}{B}{A}\mput*{\ovalnode{d}{#4}}
        \end{pspicture}
        \hskip 3.1cm}
\newcommand{\cylindre}[5]{
\pstGeonode[PointSymbol=*,PointName=#2](0,0){#1}
\pstGeonode[PointSymbol=none,PointName=none](10,0){Aamoi}
\pstGeonode[PointSymbol=none,PointName=none](#4,#3){HG}
\pstGeonode[PointSymbol=none,PointName=none](-#4,#3){HD}
\pstGeonode[PointSymbol=none,PointName=none](#4,0){BG}
\pstGeonode[PointSymbol=none,PointName=none](-#4,0){BD}
\pstLineAB{HG}{BG}
\pstLineAB{HD}{BD}

\psplot[linestyle=dashed]{-#4}{#4}{#4 2 exp x 2 exp sub .5 exp #5 mul}
\psplot{-#4}{#4}{#4 2 exp x 2 exp sub .5 exp -#5 mul}
\psplot{-#4}{#4}{#4 2 exp x 2 exp sub .5 exp #5 mul #3 add}
\psplot{-#4}{#4}{#4 2 exp x 2 exp sub .5 exp -#5 mul #3 add}
}

\newcommand{\cone}[7]{
\pstGeonode[PointSymbol=*,PointName=#2](0,0){#1}
\pstGeonode[PointSymbol=none,PointName=none](10,0){Aamoi}
\pstGeonode[PointSymbol=none,PointName=none](0,#3){Hsommet}
\pstMiddleAB[PointSymbol=none,PointName=none]{#1}{Hsommet}{Milieuu}
\pstInterCC[PointSymbol=none,PointName=none,RadiusA=\pstDistVal{#6}]{#1}{}{Milieuu}{Hsommet}{Ibidon}{Jbidon}
\pstProjection[PointSymbol=none,PointName=none]{#1}{Aamoi}{Jbidon}{Jbidon'}
\pstProjection[PointSymbol=none,PointName=none]{#1}{Aamoi}{Ibidon}{Ibidon'}

\psplot[linestyle=dashed]{-#6}{#6}{#6 2 exp x 2 exp sub .5 exp #7 mul}
\psplot{-#6}{#6}{#6 2 exp x 2 exp sub .5 exp -#7 mul}
\pstHomO[HomCoef=#7,PointSymbol=none,PointName=#5]{#1}{Hsommet}{#4}
\pstHomO[HomCoef=#7,PointSymbol=none,PointName=none]{Jbidon'}{Jbidon}{T1}
\pstHomO[HomCoef=#7,PointSymbol=none,PointName=none]{Ibidon'}{Ibidon}{T2}
\pstLineAB{#4}{T1}
\pstLineAB{#4}{T2}
}

\newcommand{\rapporteur}{
\SpecialCoor
\pscircle{5}
\multido{\i=0+1}{181}{\psline(5;\i)}
\pscircle[fillcolor=white,fillstyle=solid,linestyle=none]{4.5}
\multido{\i=0+10}{19}{\psline(5;\i)}
\pscircle[fillcolor=white,fillstyle=solid,linestyle=none]{4}
\pscircle[fillcolor=white,fillstyle=solid]{3}
\psframe[fillcolor=white,fillstyle=solid,linestyle=none](5,-5.1)(-5,-1.75)
\pscircle[fillcolor=white,fillstyle=solid]{.5}
\psframe[fillcolor=white,fillstyle=solid](-5,-1.75)(5,0)
\psline[linecolor=white](-.5,0)(.5,0)
\psline[linecolor=white](-3,0)(-4,0)
\psline[linecolor=white](3,0)(4,0)
\pscircle[fillcolor=white,fillstyle=solid]{.05}
\NormalCoor
}

\newcommand{\QCM}[4]{
    \begin{tabular}[t]{p{13cm}c}
    #1 & \psset{xunit=1 cm}
    \begin{pspicture}(-0.3,0)(1.5,0.5)
    \pspolygon(0,0)(1.5,0)(1.5,-.5)(0,-.5)
    \psline(.5,0)(.5,-.5) \psline(1,0)(1,-.5)
    \uput[90](0.25,0){A}  \uput[90](0.75,0){B} \uput[90](1.25,0){C}
    \end{pspicture} \\
    A: #2 \qquad B: #3 \qquad C: #4 & \\
    \end{tabular}}

\newcommand{\smallbox}{
    \begin{pspicture}(.5,.5)
    \pspolygon(0,0)(.25,0)(.25,.25)(0,.25)
    \end{pspicture}}

\renewcommand{\vec}[1]
{\mathord{\setbox0\hbox{$#1$} \mathop{\smash{#1}\setbox1\copy0\ht1 0.8\ht0
\vphantom{\copy1}\mskip0.8\thinmuskip}
\limits^{\hbox to\wd0{$\mskip0.8\thinmuskip$\rightarrowfill}}\mskip-0.8\thinmuskip}}

\newcommand{\system}[2]{
\left\{
\begin{array}{l}
#1\\
#2
\end{array}
\right.
}

\newcolumntype{I}{!{\vrule width 1.5pt}}
\newlength\savedwidth
\newcommand\whline{\noalign{\global\savedwidth\arrayrulewidth\global\arrayrulewidth
1.5pt} \hline \noalign{\global\arrayrulewidth\savedwidth}}
% Marges
\geometry{ hmargin=0.5cm, vmargin=1.5cm }
\usepackage{mathrsfs}   % Police de maths joli caligraphie
\newcommand{\Calig}[1]{\ensuremath{\mathscr{#1}}}
% Exo
\newcommand{\MonRepereij}[1][3pt 3]{%
	\psline[arrowsize=#1,linewidth=1.2pt]{<->}(1,0)(0,0)(0,1)
	\uput{.2cm}[270](0.5,0){$\vec{\imath}$}
	\uput{.25cm}[180](0,0.5){$\vec{\jmath}$}}
	\newcommand{\vecteur}[1]{%
	\overrightarrow{\rule{0em}{1.8ex} #1 \rule{0.15em}{0ex}}}
\newcounter{nexo}
\setcounter{nexo}{0}
\newcommand{\exo}{
    \stepcounter{nexo}
    {\textbf{$\triangleright$ \underline{Exercice \arabic{nexo} :}}}
}% Questions
\newenvironment{questions}{\begin{enumerate}[1 $\, \diamond$]}{\end{enumerate}}

    \lhead{\textsf{Lycée } - \textit{NSI}}
    \chead{Dictionnaires et recherche en table}
    \rhead{\textit{Année} 2019/2020}
    \rfoot{\textsf{Aude Duhem, Sophie Duvauchelle, Patrice Nicolas}}
    \pagestyle{fancy}
  \renewcommand{\headrulewidth}{0.5pt}

%%%%%%%%%%%%%%%%%%%%%%%%%%%%%%%%%%%%%%%%%%%%%%% Début du document %%%%%%%%%%%%%%%%%%%%%%%%%%%%%%%%%%%%%%

\begin{document}
%\begin{landscape}
\begin{center}
\large\textbf{Les dictionnaires et recherche en table - feuilles d'exercices  - Compléments pédagogiques}
\end{center}
 \hrule\vspace{\baselineskip}
 \normalsize
 Pré-requis : chaines de caractères,listes, boucles for et while, test : if ..., booléens, fonctions\\


intérêt :\\
$\ast$ exercices variés : débranché ou pas\\
$\ast$ difficulté différente : de l'application directe, à l'apprentissage vers l'approfondissement\\
$\ast$ plus ou moins guidé \\

Compétence attendues :\\
$\ast$ apprentissage et approfondissement type dictionnaire, méthode par compréhension\\
$\ast$ approfondissement de la notion de liste\\


%%%%%%%%%%%%%%%%%%%%%%%%%%%%%%%%%%%%%%%%%%%%%%%% DEBUT DES EXERCICES %%%%%%%%%%%%%%%%%%%%%%%%%%%%%%%%%%
\hrule
\hrule
\begin{center} Les deux premiers exercices portent sur les premières notions du cours (page 1)\end{center}
\exo   exercice débranché\\
Découverte des premières notions  :\\
$\bullet$ des façons "simples" de déclarer/construire un objet type dictionnaire : d'une seul coup avec les accolades, par ajout un à un des élements et la fonction dict.\\
$\bullet$ modification d'un dictionnaire : ajout/suppression d'un couple cle/valeur\\
$\bullet$ accès à une valeur et les types d'erreurs potentielles KeyError ou pas d'index\\
$\bullet$ taille de dictionnaire\\
avec deux types de questions :  QCM  et imaginer les réponses obtenues dans la boucle d'interaction\\
\hrule
\exo exercice débranché puis branché portant sur :\\
la création et la modification d'un dictionnaire et l'accès à une valeur\\
Pas d'écriture de fonction dans cet exercice afin de bien apréhender le type dictionnaire\\
Une question facultative pour introduire la recherche d'une clef dans un dictionnaire et le parcours d'un dictionnaire (deuxième partie du cours)
 \hrule
 \hrule
 \begin{center}  Les exercices 3 à 4 portent sur les manipulations de dictionnaires \end{center}
 \exo\\
 Exercice portant sur le parcours et l'affichage d'un dictionnaire : recours à l'opérateur in, aux boucles for et à un test \\
 \hrule
 \exo\\
 Exercice ultra guidé pour utiliser les notions, opérateurs et principales méthodes des dictionnaires\\
 création de deux dictionnaires, écritures de fonctions faisant appel à methodes keys, values, items, à l' opérateur in et à la fonction len et à des tests\\
 premières recherches dans des dictionnaires\\
 \hrule
  \hrule
 \begin{center}  L'exercice 5 porte sur la notion de compréhension.\end{center}
 \exo Cette notion pouvant s'avérer délicate, la notion est dans un premier temps en débranché.\\
Cet exercice a pour but de consolider cette notion afin que les élèves arrivent, seuls, à créer des dictionnaires par compréhension dans l'exercice 3 du TP\\
 \hrule
 \begin{center}  L'exercice 6 est un exercice bilan un peu moins guidé.\end{center}
 \exo  Exercice permettant de faire la synthèse des points principaux et éventuellement de faire de la rémédiation en fonction des résultats de l'évaluation. (il sera dans ce cas un peu modifié)
 
\end{document}