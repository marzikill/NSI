%! TEX program = xelatex
% Nécessaire pour la compilation pstricks
\documentclass[12pt]{article}                   % autres choix : report, book
\usepackage[a4paper]{geometry}                  % taille correcte du papier
\usepackage{silence}                            % Filtrer tous ces warnings ennuyeux
\usepackage[utf8]{inputenc}                     % encodage du fichier source
\usepackage[T1]{fontenc}                        % gestion des accents (pour les pdf)
\usepackage[french]{babel}                    % rajouter éventuellement english, greek, etc.
\usepackage{eurosym}                            % symbole euro

\usepackage{textcomp}                           % caractères additionnels
\usepackage{lmodern}                            % remplacer éventuellement par txfonts, fourier, etc.
\usepackage{amsmath,amssymb, amsthm}            % tout ce qu'il faut pour écrire des maths
\usepackage{dsfont}                             % Pour la fonction indactrice
\usepackage{xfrac}                              % fraction d'anneau

\usepackage{adforn}                             % Ornements header
\usepackage{yfonts}                             % Fontes gothiques supplémentaires
\usepackage{fourier-orns}                       % Ornements header
\usepackage{mathpazo}
\usepackage{pstricks,pst-plot,pstricks-add}     % tracer des fonctions
\usepackage[linesnumbered, french]{algorithm2e} % mise en page des algorithmes
\usepackage{tikz, tkz-euclide, tkz-tab}         % Dessiner aver tikz
\usetkzobj{all}                                 % et tout plein de paquets utiles
\usetikzlibrary{arrows.meta}
\usetikzlibrary{automata,positioning}
\usetikzlibrary{decorations.pathreplacing,calc}
\usetikzlibrary{arrows}
\usetikzlibrary{calc}
\usetikzlibrary{decorations.fractals}           % Dessiner les fractales usuelles
\usepackage{enumitem}                           % 1, 2, 3...
\usepackage{pgffor}                             % Pour faire des boucles

\usepackage{multicol}
\usepackage{comment}
\usepackage{array}                              % tableaux un peu plus paramétrables

\usepackage{minted}
\usepackage[makestderr]{pythontex}

\usepackage{graphicx}                           % pour inclure des images
\usepackage{xspace}                             % protection des espaces
\usepackage{xcolor}                             % pour gérer les couleurs
\usepackage{microtype}                          % améliorations typographiques
\usepackage{hyperref}                           % gestion des hyperliens
\hypersetup{
    colorlinks = true,
    linkcolor = blue!50!gray!100!,
    linkbordercolor = {white},
}
\hypersetup{pdfstartview=XYZ}                   % Zoom hyperref
 
%%%%%%%%%%%
%% Maths %%
%%%%%%%%%%%
% Raccourcis pour les trucs longs à écrire. Reste d'Oxford principalement. 
\newcommand{\Prob}[2][]{\mathbb{P}_{#1}\left( #2 \right)}
\newcommand{\Exp}[2][]{\mathbb{E}_{#1}\left[ #2 \right]}
\newcommand{\Lawconv}{\stackrel{\mathcal{L}}{\longrightarrow}}
\newcommand{\Var}[1]{\mbox{Var}\left(#1 \right)}
\newcommand{\Card}[1]{\mbox{Card}\left(#1 \right)}
\newcommand{\Int}[4]{\int_{#1}^{#2} \! {#3} \, \mathrm{d}{#4}}
\newcommand{\kw}{\;|\;}
\newcommand{\im}{\textrm{Im}}
\newcommand{\disclamer}{\textit{L’usage de toute calculatrice, téléphone portable, ordinateur est strictement interdit. Un soin tout particulier devra être porté à la qualité et à la précision de la rédaction des argument } \vspace{1.cm}}
\newcommand{\legendeexos}{
\paragraph{Légende des exercices.}
\begin{center}
\begin{tabular}{cccc}
$\clubsuit$ : très flacile & $\varheart$ : à savoir par cœur & $\vardiamond$ : care aux calculs & $\spadesuit$ : qui s'y frotte s'y pique 
\end{tabular}
\end{center}
}

%%%%%%%%%%%%%%%%
%% Structures %%
%%%%%%%%%%%%%%%%

\newcounter{dfcount}
{\theoremstyle{definition} \newtheorem{definition}[dfcount]{Définition}}
\newcounter{propcount}
\newtheorem{propriete}[propcount]{Propriété}
\newcounter{thcount}
\newtheorem{theoreme}[thcount]{Théorème}
\newcounter{excount}
{\theoremstyle{definition} \newtheorem{exemple}[excount]{Exemple}}
\renewcommand{\theexemple}{\arabic{exemple}}
% Environnement exos
\newcounter{exoscount}
\newtheoremstyle{exercicestyle} % name of the style to be used
  {20pt}                        % measure of space to leave above the theorem. E.g.: 3pt
  {3pt}                         % measure of space to leave below the theorem. E.g.: 3pt
  {}                            % name of font to use in the body of the theorem
  {}                            % measure of space to indent
  {\bfseries}                   % name of head font
  {.}                           % punctuation between head and body
  {.5em}                        % space after theorem head; " " = normal interword space
  {}                            % Manually specify head
\theoremstyle{exercicestyle}
\newtheorem{exos}[exoscount]{Exercice}
\newenvironment{exo}[2]
  {
   \begin{exos}[#1]
   \leavevmode
   \marginpar{\hfill $#2 $ }}
  {\end{exos}}

\newenvironment{prob}{ \begin{center}
\Large{ \textsc{Problème}} \end{center} }{ \sepdown }

 % Théorèmes/propriétés avec un retour à la ligne.
\newtheoremstyle{break2}
  {\topsep}%   Space before theorem
  {\topsep}%   Space after theorem
  {\itshape}%  Body font
  {}%          Indent amount (empty = no indent, \parindent = para indent)
  {\bfseries}% Thm head font   
  {}%         Punctuation after thm head
  {.5em }%  Space after thm head (\newline = linebreak)
  {\thmname{#1}\thmnumber{ #2}\thmnote{ (#3)}.}% Thm head spec

\theoremstyle{break2}
\newtheorem{fbth}[thcount]{Théorème}
\newenvironment{bth}[1][]{\begin{fbth}[#1]\item\relax}{\end{fbth}}

\newtheorem{fbprop}[propcount]{Propriété}
\newenvironment{bprop}[1][]{\begin{fbprop}[#1]\item\relax}{\end{fbprop}}

 % Théorèmes/propriétés avec un retour à la ligne.
\newtheoremstyle{break3}
  {\topsep}%   Space before theorem
  {\topsep}%   Space after theorem
  {}%  Body font
  {}%          Indent amount (empty = no indent, \parindent = para indent)
  {\bfseries}% Thm head font   
  {}%         Punctuation after thm head
  {.5em }%  Space after thm head (\newline = linebreak)
  {\thmname{#1}\thmnumber{ #2}\thmnote{ (#3)}.}% Thm head spec

\theoremstyle{break3}
\newtheorem{fbdf}[dfcount]{Définition}
\newenvironment{bdef}[1][]{\begin{fbdf}[#1]\item\relax}{\end{fbdf}}

% Répéter un exercice n fois
\newcommand{\looped}[2]{
	\foreach \n in {1,...,#1}{
		#2
	}
}
%%%%%%%%%%%%%%%%%%
%% Mise en page %%
%%%%%%%%%%%%%%%%%%

\setlength{\textheight}{26.2cm}
\setlength{\textwidth}{17.0cm}
\setlength{\oddsidemargin}{-0.5cm}
\setlength{\evensidemargin}{0.0cm}
\setlength{\topmargin}{-2.0cm}
\pagestyle{empty}
\newif \ifnote
\notefalse
\reversemarginpar

\usepackage{titlesec}      % Interlignes pas trop gros
\titlespacing{\paragraph}{ %
  0pt}{                    % left margin
  1\baselineskip}{         % space before (vertical)
  1em}                     % space after (horizontal)

\usepackage{titlesec}
%\titleformat
%   {\section} % <command>
%   {\normalfont\Large\bfseries} % <format>
%   {\thesection} % <label>
%   {1em} % <sep>
%   {} % <before-code>
%\titleformat{\section}{\bfseries\sffamily\large}{\thesection.}{\hspace{1cm}}{}%
\titlespacing*                  % starred version: first paragraph is not indented
    {\section}                  % <command>
    {-22pt}                     % <left>
    {3.5ex plus 1ex minus .2ex} % <before-sep>
    {2.3ex plus .2ex}           % <after-sep>


\setlength{\columnseprule}{1pt}        % Séparateur vertical dans multicol
\def\columnseprulecolor{\color{black}} % De couleur noire.

% Type de colonnes pour array
\newcolumntype{P}[1]{>{\centering\arraybackslash}p{#1}}
\newcolumntype{M}[1]{>{\centering\arraybackslash}m{#1}}

% Filtrer les messages d'erreur kisoulent 
\WarningFilter{latex}{Marginpar on page}
\WarningFilter{latexfont}{Size}

%%%%%%%%%%%%%%%%%%%%%%%%%
%% Faire du joli latex %% 
%%%%%%%%%%%%%%%%%%%%%%%%%

% Faire des cœurs et des carreaux 
\DeclareSymbolFont{extraup}{U}{zavm}{m}{n}
\DeclareMathSymbol{\varheart}{\mathalpha}{extraup}{86}
\DeclareMathSymbol{\vardiamond}{\mathalpha}{extraup}{87}

% Police verbatim + souligné (links)
\usepackage[normalem]{ulem}
\useunder{\uline}{\ulined}{}%
\DeclareUrlCommand{\bulurl}{\def\UrlFont{\ttfamily\ulined}}


\newcommand{\expandpyconc}[1]{\expandafter\reallyexpandpyconc\expandafter{#1}}
\newcommand{\reallyexpandpyconc}[1]{\pyconc{exec(compile(open('#1', 'rb').read(), '#1', 'exec'))}}

\newenvironment{pyconcodeblck}[1]%
 {\newcommand{\snippetfile}{snippet-#1.py}
  \VerbatimEnvironment
  \begin{VerbatimOut}{\snippetfile}}%
 {\end{VerbatimOut}%
  \expandpyconc{\snippetfile}}

\begin{document}
%%%%%%%%%%%%%%%%%%%%%%
%% Header tout joli %%
%%%%%%%%%%%%%%%%%%%%%% 
\noindent\hrulefill
\raisebox{-2.1pt}[10pt][10pt]{\quad\decoone\quad}\hrulefill 

\leftline{$\textgoth{L}$ycée \hfill $\textgoth{I}$nformatique}
\hfill $\textgoth{S}$pécialité
 \begin{center}{\adforn{64}\quad \large{\textbf{ Représentation des données : les tableaux}}\; \adforn{36}}\end{center}
\vspace{0.5cm}
\noindent
\hrulefill
\raisebox{-16.4pt}[10pt][10pt]{\begin{tikzpicture}[decoration=Koch snowflake]
\draw decorate{ decorate{ decorate{ decorate{ (6,-0) -- (4,0) -- (2,0) -- (0,0)  }}}};
\end{tikzpicture}}\hrulefill
\vspace{1cm}

%%%%%%%%%%%%%%%%%%%%
%% Corps du texte %%
%%%%%%%%%%%%%%%%%%%%
% \disclamer pour les examens
% qcours<Tab> pour les colles
% \legendeexos pour la légende

\section{Définition et exemples}

\begin{definition}
Un \textbf{tableau} (le mot clé correspondant en Python est \texttt{list}) est un objet informatique (une \textbf{structure de données}) qui permet de stocker plusieurs valeurs.  
\end{definition} 

\begin{exemple}
    \label{ex:tableau_init}
 Au premier trimestre, Sarah a obtenu les notes suivantes : 10, 15, 11, 18, 5, et 20. Si on souhaite stocker ces notes dans des variables, on doit donc écrire :
\inputminted[frame=single, framesep=2mm, label=Code Source,linenos=true]{python}{./pyfiles/notes.py}
En prenant exemple sur les tableurs comme LibreOffice Calc, on pourrait plutôt décider de stocker ces variables dans un tableau :
\begin{center}
\begin{tabular}{|M{3cm}|c|c|c|c|c|c|c|}
    \hline
    Notes de Sarah & 10 & 15 & 11 & 12 & 18 & 5 & 20 \\
    \hline
\end{tabular}
\end{center}

En Python nous allons déclarer une variable \texttt{notes\_sarah} de type tableau qui stocke les notes de Sarah. On utilise pour cela les symboles \textbf{crochet '[' et ']'}, et on sépare les éléments du tableau par des \textbf{virgules}.

\begin{pyconsole}[][frame=single, framesep=2mm, label=Console Python,linenos=true]
notes_sarah = [10, 15, 11, 12, 18, 5, 20]
notes_sarah[0]
notes_sarah[3]
\end{pyconsole}

C'est beaucoup plus court ! Allons voir maintenant comment créer et manipuler les tableaux en Python.
\end{exemple}

\pagebreak

\section{Création et manipulation de tableaux}

\subsection{Création d'un tableau}

\begin{definition}
    Pour créer un tableau en Python, trois méthodes sont possibles :
    \begin{itemize}[label=\textbullet]
        \item par \textbf{énumération} : on définit le tableau par la liste de ses valeurs, comme dans l'exemple \ref{ex:tableau_init} :
\begin{pyconsole}[][frame=single, framesep=2mm, label=Console Python,linenos=true]
notes_sarah = [10, 15, 11, 12, 18, 5, 20]
print(notes_sarah)
\end{pyconsole}
        \item par \textbf{initialisation} : on définit un tableau de \texttt{n} cases, dont tous les contenus sont nuls :
\begin{pyconsole}[][frame=single, framesep=2mm, label=Console Python,linenos=true]
n = 10
tableau = [0]*n
print(tableau)
\end{pyconsole}
        \item par \textbf{ajouts successifs} : on rajoute les éléments au tableau les uns après les autres (à la main ou avec une boucle) :
\begin{pyconsole}[][frame=single, framesep=2mm, label=Console Python,linenos=true]
tableau = []
tableau.append(5)
tableau.append(120)
print(tableau)
\end{pyconsole}
    \end{itemize}
\end{definition} 

\paragraph{Remarques.} Il existe un objet particulier, le \textbf{tableau vide} : c'est un tableau qui ne contient aucune valeur.

\textit{"append"} signifie \textit{"ajouter à la fin"}. Que se serait-il passé si on avait inversé les lignes 2 et 3 du code correspondant aux ajouts successifs ?

\begin{definition}
    La \textbf{taille} d'un tableau est le nombre d'éléments qui le composent. En Python, \texttt{len(tableau)} renvoit la taille de \texttt{tableau}.
\end{definition} 

\begin{exemple}
   Quelle est la taille des tableaux suivants ?
   \begin{itemize}[label=\textbullet]
       \item \texttt{notes\_sarah = [10,15,11,12,18,5,20]} a pour taille 7.
       \item \texttt{[4,5,2,3,8,5,65,8]} a pour taille : \dotfill
       \item \texttt{[0]*123} a pour taille : \dotfill
       \item  \texttt{[]} a pour taille : \dotfill
   \end{itemize}
\end{exemple}


\subsection{Manipulation d'un tableau}

\begin{definition}
    Soit \texttt{t} un tableau. Les éléments du tableau sont ordonnés et chaque valeur du tableau est repérée par un unique \textbf{indice}. On accède à chaque élément du tableau par son indice.
\end{definition} 

\paragraph{ATTENTION.} En Python, les indices commencent à 0.

\begin{exemple}
    Reprenons l'exemple des notes de Sarah.

\begin{center}
\begin{tabular}{M{3cm}ccccccc}
    \texttt{notes\_sarah =} & \texttt{[10,} & \texttt{15,} & \texttt{11,} & \texttt{12,} & \texttt{18,} & \texttt{5,} & \texttt{20]} \\
    Indice & 0 & 1 & 2 & 3 & 4 & 5 & 6 \\
\end{tabular}
\end{center}

Pour accéder à la cinquième note (18/20) de Sarah :

\begin{pyconsole}[][frame=single, framesep=2mm, label=Console Python,linenos=true]
notes_sarah = [10, 15, 11, 12, 18, 5, 20]
notes_sarah[4]
\end{pyconsole}
\end{exemple}

\begin{exo}{}{  }
    On considère le tableau suivant :
\begin{center}
\begin{tabular}{M{3cm}ccccccccc}
    \texttt{test =} & \texttt{[555,} & \texttt{951,} & \texttt{-159,} & \texttt{258,} & \texttt{852,} & \texttt{-456,} & \texttt{654,}  & \texttt{123,} & \texttt{987]}\\
\end{tabular}
\end{center}
\begin{enumerate}
    \item Quelle est la taille du tableau ? \dotfill
    \item Quel est l'élément d'indice 2 ? \dotfill
    \item Quel est l'indice de 852 ? \dotfill
    \item Quel est l'indice du dernier élément ? \dotfill
\end{enumerate}

\end{exo} 

\paragraph{Erreur d'indice.} Lorsque l'on demande un indice trop grand pour la taille du tableau, Python renvoie une erreur appelée \texttt{IndexError}.

\begin{pyconsole}[][frame=single, framesep=2mm, label=Console Python,linenos=true]
notes_sarah = [10, 15, 11, 12, 18, 5, 20]
notes_sarah[10]
\end{pyconsole}

Si \texttt{t} est un tableau, alors les indices compatibles avec ce tableaux sont tous les entiers compris entre 0 et la taille du tableau moins 1.

\paragraph{Remarque.} L'indice du dernier élément du tableau est la taille du tableau \textbf{moins 1}.

\paragraph{Modification des éléments d'un tableau.} Si \texttt{t} est une variable de type tableau, on peut modifier les valeurs de ses éléments à l'aide de l'indice correspondant :

\begin{pyconsole}[][frame=single, framesep=2mm, label=Console Python,linenos=true]
notes_sarah = [10, 15, 11, 12, 18, 5, 20]
notes_sarah[5]
notes_sarah[5] = 15
print(notes_sarah)
\end{pyconsole}

\begin{exo}{}{  }
    Le professeur a été trop sévère avec les élèves dans sa notation. Il décide donc d'augmenter les notes des premier et troisième contrôle de 1 point. 

    Écrire une fonction \texttt{mise\_a\_jour} qui prend en entrée un tableau de notes \texttt{t} de taille supérieure ou égale à 3 et renvoie en sortie le tableau des notes mises à jour. Vous testerez votre fonction :

    \begin{center}
        \texttt{mise\_a\_jour([5, 7, 6, 9, 10, 11, 10])} renvoie \texttt{[6, 7, 7, 9, 10, 11, 10]}. 
    \end{center}
\end{exo} 

\paragraph{Correction.}
\inputminted[frame=single, framesep=2mm, label=Code Source,linenos=true]{python}{./pyfiles/mise_a_jour.py}

\end{document}
 
