% !TeX document-id = {8613a224-84c8-4428-9607-d4a0352535f7}
%pdflatex -shell-escape doc4.tex
%pythontex doc4.tex
%pdflatex -shell-escape doc4.tex
% !TeX TXS-program:compile = txs:///pdflatex/[--shell-escape]
\documentclass[a4paper,12,answers]{exam}
\usepackage{fourier}
\usepackage{geometry}
\geometry{vmargin=10mm,hmargin=5mm}
\usepackage{fourier}
\usepackage[utf8]{inputenc}
\usepackage{geometry}
\geometry{vmargin=10mm,hmargin=5mm}
%\usepackage[latin1]{inputenc}
\usepackage[T1]{fontenc}
\usepackage[francais]{babel}
\usepackage{listings}
\usepackage{pythontex}
\usepackage{minted}
%renewcommand\listingscaption{Code Source}
\usepackage[table]{xcolor}
\usepackage{graphicx}
\graphicspath{ {/Users/the_brat/Documents/Mon_Latex} }
\usepackage{tabto}
%\usepackage{minibox}
\definecolor{ivory}{rgb}{1.0, 1.0, 0.94}
\definecolor{cadetgrey}{rgb}{0.57, 0.64, 0.69}
\usepackage{tikz,lipsum,lmodern}
\usepackage[most]{tcolorbox}

\usepackage[titletoc]{appendix}
\usepackage{wrapfig}
%\usepackage[many]{tcolorbox}%https://tex.stackexchange.com/questions/262052/lstlisting-framed-with-the-caption-inside-the-frame
%\tcbuselibrary{listings}


\newcommand{\expandpyconc}[1]{\expandafter\reallyexpandpyconc\expandafter{#1}}
\newcommand{\reallyexpandpyconc}[1]{\pyconc{exec(compile(open('#1', 'rb').read(), '#1', 'exec'))}}

   

\begin{document}
\lhead{Lycée Jean Monnet - \textit{NSI}}
\chead{}
\rhead{\textit{Année} 2019/2020}

\lfoot{                      }\cfoot{Page \thepage}\rfoot{\textsf{Aude Duhem, Sophie Duvauchelle, Patrice Nicolas}}
\hrule
\begin{center}
	\textbf{\Large{QCM - Les dictionnaires et recherche en table }}\end{center}
\hrule
\vskip0.5cm
	\begin{questions}
		\question On désire enregistrer les notes ci-sessous d'un élève par matière dans un objet de type dictionnaire : 
		\begin{center}
		\begin{tabular}{|c|c|c|c|}
		\hline
		Matière (Clé) & "maths"&"anglais"&"sport"\\
		\hline
		 Liste de notes (Valeur)&[13,15] &[16,12,14]&[17]\\
		 \hline
		\end{tabular}
	\end{center}
		Quelle est la succession d'instructions qui ne convient pas ?
		Cocher la bonne réponse:
		\begin{checkboxes}
			\choice Réponse A : 
\begin{pyverbatim}
Notes={}
Notes['maths']=[13,15]
Notes['anglais']=[16,12,14]
Notes['sport']=[17]
\end{pyverbatim}
\choice Réponse B :
\begin{pyverbatim}
Notes={'anglais': [16, 12, 14], 'sport': [17],'maths': [13, 15]}
\end{pyverbatim}
\choice Réponse C :
\begin{pyverbatim}
matieres=['maths','anglais','sport']
resultats=[[13, 15],[16, 12, 14], [17]]
Notes={}
for i in range(len(matieres)):
	Notes[i]=resultats[i]
\end{pyverbatim}
\end{checkboxes}		

\begin{solution}
C'est la réponse C, qui ne convient pas. En effet, dans la console, on obtient le résultat suivant :
\begin{pyconsole}
matieres=['maths','anglais','sport']
resultats=[[13, 15],[16, 12, 14], [17]]
Notes={}
for i in range(len(matieres)):
	Notes[i]=resultats[i]
Notes
\end{pyconsole}
\end{solution}

\question Un QCM est une liste de plusieurs questions auxquelles sont associées plusieurs réponses possibles. On peut donc coder un QCM par une liste de dictionnaires.
On dispose d'un tel questionnaire codé en Python : \\
\begin{tcolorbox}[enhanced,colback=blue!5!white,colframe=blue!75!black]	
\begin{pyverbatim}
QCM = [{'question' : "Python est un langage  ",'correcte': 'interprété', 'incorrecte':'compilé' },
		{'question' : "En Python, l'opérateur // fait ",'correcte': 'une division entière',
		 'incorrecte':'une division décimale' },
		{'question' : "En Python le mot clé utilisé pour déclarer une fonction est  ",
		'correcte': 'def', 'incorrecte':'fun' }]
\end{pyverbatim} 
\end{tcolorbox}
On désire procéder à l'affichage de toutes les questions et les réponses associées possibles. Quelle est la fonction qui convient ?
Cocher la bonne réponse:
\begin{checkboxes}
\choice Réponse A : 
\begin{pyverbatim}
def questionnaire(qs):
	"""Pose toutes les questions du questionnaire "qs" passé en argument."""
	for i in range(len(qs)):
		print("Question", i+1, ":")
		for d in  qs[i].values():
			print(d)
\end{pyverbatim}
\choice Réponse B :
\begin{pyverbatim}
def questionnaire(qs):
	"""Pose toutes les questions du questionnaire "qs" passé en argument."""
	for i in range(len(qs)):
		print("Question", i+1, ":")
		for d in  qs[i].keys():
			print(d)
\end{pyverbatim}
\choice Réponse C :
\begin{pyverbatim}
def questionnaire(qs):
	"""Pose toutes les questions du questionnaire "qs" passé en argument."""
	for i in range(len(qs)):
		print("Question", i+1, ":")
		for d in  qs[i].items():
			print(d)
\end{pyverbatim}
\end{checkboxes}		
\begin{solution}
C'est la réponse A, qui convient. En effet, dans la console, on obtient les résultats suivants :
\begin{pyconsole}
QCM = [{'question' : "Python est un langage  ",'correcte': 'interprété', 'incorrecte':'compilé' },
{'question' : "En Python, l'opérateur // fait ",'correcte': 'une division entière',
	'incorrecte':'une division décimale' },
{'question' : "En Python le mot clé utilisé pour déclarer une fonction est  ",
	'correcte': 'def', 'incorrecte':'fun' }]
#reponse A
def questionnaire(qs):
	"""Pose toutes les questions du questionnaire "qs" passé en argument."""
	for i in range(len(qs)):
		print("Question", i+1, ":")
		for d in  qs[i].values():
			print(d)

questionnaire(QCM)
#reponse B
def questionnaire(qs):
	"""Pose toutes les questions du questionnaire "qs" passé en argument."""
	for i in range(len(qs)):
		print("Question", i+1, ":")
		for d in  qs[i].keys():
			print(d)

questionnaire(QCM)	
#reponse C
def questionnaire(qs):
	"""Pose toutes les questions du questionnaire "qs" passé en argument."""
	for i in range(len(qs)):
		print("Question", i+1, ":")
		for d in  qs[i].items():
			print(d)

questionnaire(QCM)	
\end{pyconsole}
\end{solution}

\end{questions}	
\end{document}    


