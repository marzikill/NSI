%pdflatex -shell-escape doc4.tex
%pythontex doc4.tex
%pdflatex -shell-escape doc4.tex
\documentclass[a4paper,11,%answers
]{exam}
\usepackage[left=1cm, right=1cm, top=0.5cm, bottom=2cm, textwidth=10cm]{geometry}
\UseRawInputEncoding%https://www.latex-project.org/news/latex2e-news/ltnews28.pdf
%\usepackage[latin1]{inputenc}
%\usepackage[utf8]{inputenc}
\usepackage[T1]{fontenc}
\usepackage[francais]{babel}
\usepackage{listings}
\usepackage{pythontex}
\usepackage{minted}
%renewcommand\listingscaption{Code Source}
\usepackage[table]{xcolor}
\usepackage{graphicx}
\graphicspath{ {/Users/the_brat/Documents/Mon_Latex} }
\usepackage{tabto}
%\usepackage{minibox}
\definecolor{ivory}{rgb}{1.0, 1.0, 0.94}
\definecolor{cadetgrey}{rgb}{0.57, 0.64, 0.69}
\usepackage{tikz,lipsum,lmodern}
\usepackage[most]{tcolorbox}
\usepackage{amsmath}
\usepackage{amssymb}
\usepackage[titletoc]{appendix}
\usepackage{wrapfig}
%\usepackage[many]{tcolorbox}%https://tex.stackexchange.com/questions/262052/lstlisting-framed-with-the-caption-inside-the-frame
%\tcbuselibrary{listings}


\newcommand{\expandpyconc}[1]{\expandafter\reallyexpandpyconc\expandafter{#1}}
\newcommand{\reallyexpandpyconc}[1]{\pyconc{exec(compile(open('#1', 'rb').read(), '#1', 'exec'))}}

\newenvironment{pyconcodeblck}[1]%
 {\newcommand{\snippetfile}{snippet-#1.py}
  \VerbatimEnvironment
  \begin{VerbatimOut}{\snippetfile}}%
 {\end{VerbatimOut}%
  \expandpyconc{\snippetfile}}
\newcommand{\typesetcode}[1]{\inputpygments{python}{snippet-#1.py}}

\begin{pyconcodeblck}{temp3}
D1 = { 'Aude': 'Paris', 'Patrice': 'Rennes', 'Sophie': 'Lyon' }
\end{pyconcodeblck}

\begin{pyconcodeblck}{temp1}
D2={
	'1234567890G' :
	{'nom' : 'Nicolas', 'classe': '1ere A', 'specialites' : ['Maths','Physique','NSI']} ,
	'1357986420F' :
	{'nom' : 'Lucie', 'classe': '1ere B', 'specialites' : ['Maths','SES','HG']}
	}
\end{pyconcodeblck}




\lstset
{ %Formatting for code 
    language=Python,
    basicstyle=\footnotesize,
    numbers=left,
    stepnumber=1,
    showstringspaces=false,
    tabsize=1,
    breaklines=true,
    breakatwhitespace=false,}
    
\title{Exemple d'�valuation (formative) sur les dictionnaires du langage Python }
 
\begin{document}
	\maketitle
	\begin{questions}
		\question[1] Donner une d�finition de l'objet "dictionnaire" du langage Python. On rappellera la syntaxe g�n�rale.
		\begin{solution}
		En python, un dictionnaire est un tableau associatif entre des cl�s et des valeurs de sorte que chaque cl� soit unique. 
		La syntaxe est : \textbf { \{ cl�1 : valeur1, cl�2 : valeur2, ...\}}
		\end{solution}
		\question[2] On consid�re le dictionnaire suivant:\typesetcode{temp3}

\begin{parts}
\part Ecrire une commande permettant de supprimer la valeur 'Rennes'.
\begin{solution}

\begin{pyconsole}
print(D1)
del D1['Patrice']
print(D1)
\end{pyconsole}
\end {solution}
\part Qu'affiche la console si on ex�cute la commande suivante: 
\begin{pyverbatim}
del D1[ 'Isabelle' ] 
\end{pyverbatim}
\begin{solution}
\begin{pyconsole}
del D1[ 'Isabelle' ] 
\end{pyconsole}
\end {solution} 
\part Qu'affiche la console  si on ex�cute les commandes suivantes: 
\begin{pyverbatim}
D1['Aude']='Versailles'
print(D1)
\end{pyverbatim}
\begin{solution}
\begin{pyconsole}
D1['Aude']='Versailles'
print(D1)
\end{pyconsole}
\end {solution} 
 \end{parts} 

\question[1]Citer une propri�t� remarquable des cl�s d'un dictionnaire en Python.
\begin{solution}
Chaque cl� est unique.
Certains types d'objet ne peuvent pas �tre des cl�s de dictionnaires: les objet du type \textbf {list} et du type \textbf {dict} en particulier.
\end{solution}

\question[1]On consid�re l'instruction suivante permettant de cr�er l'objet L1:
\begin{pyverbatim}
L1 = [v for c,v in D1.items( ) if len(v)<6]
\end{pyverbatim} 
Cocher la (les) bonne(s) r�ponse(s):\\
\tabto{1cm}$\square$ L1 est un dictionnaire\\
\tabto{1cm}$\square$ L1 est cr�e par compr�hension\\
\tabto{1cm}$\square$ La "longueur" de L1 est �gale � 6\\
\tabto{1cm}$\square$ L1==['Paris', 'Lyon'] renvoie True\\
%\begin{checkboxes}
%\choice L1 est un dictionnaire
%\choice L1 est cr�e par compr�hension
%\choice La "longueur" de L1 est �gale � 6
%\choice L1==['Paris', 'Lyon'] renvoie True
%\end{checkboxes}		

\question[5] Consid�rons des �l�ves de classe de premi�re. On souhaite alimenter un dictionnaire  D2 dans lequel:
\begin{itemize}
\item Chaque �l�ve est identifi� par un "num�ro" unique de 11 caract�res ; ce num�ro constitue la cl� principale du dictionnaire.
\item Chaque valeur est un dictionnaire dont les cl�s sont 'nom', 'classe' et 'specialites'.
\end{itemize}
Voici le contenu du dictionnaire D2 :
\typesetcode{temp1}
\begin{parts}
\part Ecrire une ligne de commande afin d'afficher le nombre d'�l�ves pr�sents dans le dictionnaire
\begin{solution}

\begin{pyconsole}
print(len(D2))
\end{pyconsole}
\end {solution}
\part Que renvoient les commandes suivantes ?

\begin{itemize}
\item  
\begin{pyverbatim} 
D2['1234567890G'] ['Nicolas']
\end{pyverbatim}
\item  
\begin{pyverbatim} 
D2['1234567890G'] ['nom']
\end{pyverbatim}
\item  
\begin{pyverbatim} 
D2['1357986420F'] ['specialites'] [1]
\end{pyverbatim}
\end{itemize}

\begin{solution}
\begin{pyconsole}
D2['1234567890G'] ['Nicolas']
D2['1234567890G'] ['nom']
D2['1357986420F'] ['specialites'] [1]
\end{pyconsole}
\end {solution}
\part En r�alit�, Nicolas est en '1ereB' . Quelle commande peut-on ex�cuter pour mettre � jour le dictionnaire D2 ?
\begin{solution}
\begin{pyconsole}
print(D2)
D2['1234567890G'] ['classe'] = '1ereB'
print(D2)
\end{pyconsole}
\end {solution}
\end{parts}

\end{questions}	
\end{document}    


