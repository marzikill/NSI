%pdflatex -shell-escape doc4.tex
%pythontex doc4.tex
%pdflatex -shell-escape doc4.tex
\documentclass[a4paper,11,answers
]{exam}
\usepackage[left=1cm, right=1cm, top=0.5cm, bottom=2cm, textwidth=10cm]{geometry}
\UseRawInputEncoding%https://www.latex-project.org/news/latex2e-news/ltnews28.pdf
\usepackage[utf8]{inputenc}
\usepackage[T1]{fontenc}
\usepackage[frenchb]{babel}
\usepackage{listings}
\usepackage{pythontex}
\usepackage{minted}
\usepackage[table]{xcolor}
\setlength{\arrayrulewidth}{1mm}
\setlength{\tabcolsep}{18pt}
\renewcommand{\arraystretch}{2.5}
\usepackage{graphicx}
\graphicspath{ {/Users/the_brat/Documents/Mon_Latex} }
\usepackage{tabto}
\usepackage{amsmath}
\definecolor{ivory}{rgb}{1.0, 1.0, 0.94}
\definecolor{cadetgrey}{rgb}{0.57, 0.64, 0.69}
\usepackage{tikz,lipsum,lmodern}
\usepackage[most]{tcolorbox}
\shadedsolutions
%\framedsolutions
\usepackage[titletoc]{appendix}
\usepackage{wrapfig}
\usepackage{hyperref}
\hypersetup{
    colorlinks=true,
    linkcolor=blue,
    filecolor=magenta,      
    urlcolor=cyan,
}

%%%%%%%%%%%%%%%%%%%%%%%%%%%%%%%%%%%%%%%%%%%%
%https://github.com/gpoore/pythontex/issues/55
\newcommand{\expandpyconc}[1]{\expandafter\reallyexpandpyconc\expandafter{#1}}
\newcommand{\reallyexpandpyconc}[1]{\pyconc{exec(compile(open('#1', 'rb').read(), '#1', 'exec'))}}

\newenvironment{pyconcodeblck}[1]%
 {\newcommand{\snippetfile}{snippet-#1.py}
  \VerbatimEnvironment
  \begin{VerbatimOut}{\snippetfile}}%
 {\end{VerbatimOut}%
  \expandpyconc{\snippetfile}}
\newcommand{\typesetcode}[1]{\inputpygments{python}{snippet-#1.py}}

\begin{pyconcodeblck}{temp14}
def gloutonP(l,maxi):
    tmp=[x for x in l]
    tmp.sort(key=lambda tup: tup[1],reverse=True)
    valeur=0
    poids=0
    i=0
    solution=[]   
    while i<len(tmp):
        if (poids+tmp[i][1]) <=maxi:
            solution.append(tmp[i])
            poids +=tmp[i][1]
            valeur+=tmp[i][0]
        i+=1
    reponse=[1 if x in solution else 0 for x in l]        
        
    return reponse,valeur,poids
\end{pyconcodeblck}

\begin{pyconcodeblck}{temp15}
def gloutonV(l,maxi=30):
    tmp=[x for x in l]
    l.sort(key=lambda tup: tup[0],reverse=True)
    valeur=0
    poids=0
    i=0
    solution=[]
    while i<len(tmp):
        if (poids+tmp[i][1]) <=maxi:
            solution.append(tmp[i])
            poids +=tmp[i][1]
            valeur+=tmp[i][0]
        i+=1
    reponse=[1 if x in solution else 0 for x in tmp]
        
    return reponse,valeur,poids
\end{pyconcodeblck}

\begin{pyconcodeblck}{temp16}
def gloutonVPP(l,maxi=30):
    tmp=[x for x in l]
    l.sort(key=lambda tup: tup[0]/tup[1],reverse=True)
    valeur=0
    poids=0
    solution=[]
    i=0
    while i<len(tmp):
        if (poids+tmp[i][1]) <=maxi:
            solution.append(tmp[i])
            poids +=tmp[i][1]
            valeur+=tmp[i][0]
        i+=1
    reponse=[1 if x in solution else 0 for x in tmp]
        
    return reponse,valeur,poids

\end{pyconcodeblck}

\begin{pyconcodeblck}{temp17}
def possibilites(n):
    assert n<52 #retourne une erreur si n>=52
    alphabet=['A','B','C','D','E','F','G','H','I','J','K','L','M','N',
              'O','P','Q','R','S','T','U','V','W','X','Y','Z','a','b',
              'c','d','e','f','g','h','i','j','k','l','m','n','o','p',
              'q','r','s','t','u','v','w','x','y','z']
    tab=[]
    for i in range(n):
        tab.append(alphabet[i])
    combinaisons = ['']#initialise avec le sac vide; cas où l'objet le plus léger est plus lourd que le poids maximal
    for loop in range(n):
        tmp = [i for i in tab]
        for boucle in range(loop):
            #on compare l'élément à ajouter i avec le dernier élément ajouté x[-1]
            #on concatène des chaines
            tmp = [x+i for i in tab for x in tmp if ord(i)>ord(x[-1])]        
        combinaisons += tmp
    return len(combinaisons)#on retourne la longueur du tableau contentant toutes les combinaisons
\end{pyconcodeblck}
\begin{pyconcodeblck}{temp19}
def extraire(t,n):
    for i in range(1,len(t)):
        v = t[i][n]
        j = i-1
        while(j>=0 and t[j][n]>v):
            t[j+1],t[j] = t[j],t[j+1]
            j = j-1
        t[j+1][n] = v
    return t
\end{pyconcodeblck}
\renewcommand{\thesection}{\Roman{section}}
\renewcommand{\thesubsection}{\Alph{subsection}}
\renewcommand{\thesubsubsection}{\arabic{subsubsection}}
\title{Correction du TP n°2: Le problème du Sac à Dos}


%%%%%%%%%%%%%%%%%%%%%%%%%%%%%%%%%%%%%%%%%%%%
\usepackage{lmodern}
 


\begin{document}
\maketitle
%\setcounter{chapter}{1}
%\chapter{Un exemple d'optimisation - Les algorithmes "gloutons"}


\section{Retour sur les Tris}
 
 \begin{questions}

\question Ecrire une fonction \textbf{extraire} qui accepte en paramètres, une liste de liste  \textbf{l} et un entier \textbf{n}  dans [0; 2] .La fonction doit retourner la liste triée en fonction du n-ième  élément de chaque sous-liste. On pourra choisir le tri insertion par exemple.
 \vspace{5mm}
\begin{tcolorbox}[enhanced,attach boxed title to top center={yshift=-3mm,yshifttext=-1mm},
  colback=green!5!white,colframe=green!75!black,colbacktitle=green!25!black,
  title=Console Python, fonttitle=\bfseries,
  boxed title style={size=small,colframe=blue!25!black} ]
\begin{pyconsole}
L= [ [1, 5, 6], [8, 10, 2], [3, 3, 5],[4, 8, 1] ]
extraire(L,0)
extraire(L,1)
extraire(L,2)
\end{pyconsole}
\end{tcolorbox}

\begin{listing}[h]
\begin{minted}
[frame=leftline,
framesep=2mm,
baselinestretch=1.2,
bgcolor=,
fontsize=\footnotesize,
linenos]{python}
def extraire(t,n):
    for i in range(1,len(t)):
        v = t[i][n]
        j = i-1
        while(j>=0 and t[j][n]>v):
            t[j+1],t[j] = t[j],t[j+1]
            j = j-1
        t[j+1][n] = v
    return t

 \end{minted} 
\end{listing}

\question Pourquoi le code précédent ne fonctionne-t-il pas (tel quel) avec une liste de tuples?
\begin{solution}
Le tuple est un objet que l'on ne peut pas modifier donc une instruction comme celle de la ligne 8 par exemple n'est pas autorisée. Il faudrait par exemple, copier la liste de tuples en liste de liste puis trier et retourner une liste de tuples.
\end{solution}
\question A l'aide de l'exercice 7, expliquer le sens de l'instruction \mintinline{python}{List.sort(key = lambda a : a[1]) }

\begin{solution}
 "list.sort() and sorted() have a key parameter to specify a function to be called on each list element prior to making comparisons." [...]accept a reverse parameter with a boolean value. This is used to flag descending sorts.\newline

 Dès lors, on peut appeler une fonction lambda(nom donné à une fonction nommée "à la volée") pour trier des objets à un certain indice dans l’élément courant (si cet élément est une liste, un tuple,...) En conséquence, le tri sur a[1] signifie que l’on s’intéresse au deuxième élément du tuple. Par ailleurs, reverse=True indique qu’il s’agit d’un tri "décroissant".

\end{solution}
\section{Algorithme du sac à dos}
\subsection{Critère de Poids} 
Nous allons écrire un premier algorithme glouton de critère \textbf{"placer d'abord dans le sac, les objets les plus lourds."} En cas d'égalité, on prendra "au hasard" un objet parmi les indécis. On pourra utiliser la méthode \textbf{list.sort()} comme définie sur la documentation officielle \href{https://docs.python.org/fr/3/howto/sorting.html}{Python}
\question Ecrire une fonction \textbf{gloutonP} qui accepte deux paramètres \textbf{l} et \textbf{maxi}, respectivement une liste de tuples (valeur en €, poids en kg) et un entier correspondant au poids maximal supporté par le sac. La fonction doit donc retourner un triplet \textbf{reponse, valeur, poids} comme dans l'exemple ci-dessous:
 \vspace{5mm}
\begin{tcolorbox}[enhanced,attach boxed title to top center={yshift=-3mm,yshifttext=-1mm},
  colback=green!5!white,colframe=green!75!black,colbacktitle=green!25!black,
  title=Console Python, fonttitle=\bfseries,
  boxed title style={size=small,colframe=blue!25!black} ]
\begin{pyconsole}
liste=[(7,13), (4,12), (3,8), (3,10)]
print(gloutonP(liste,30))
\end{pyconsole}
\end{tcolorbox}
 \vspace{5mm}
\begin{solution}
(voir Annexe B) On crée une copie de la liste de tuples \textbf{lig(2 et 3)}  puis on trie la liste "temporaire" selon le 2e élément (tup[1]) dans l'ordre décroissant (reverse=True) ;  \textbf{lig(8)} on rentre dans une boucle "de la longueur" du tableau. Si le poids actuel du sac plus le poids du prochain objet est inférieur ou égal au poids maxi \textbf{lig(9)}, alors on ajoute cet objet à la solution \textbf{lig(10)}, puis on augmente le poids et la valeur du sac \textbf{lig(11 et 12)}. Enfin, on réalise une compréhension en utilisant la liste initiale pour afficher soit 1 soit 0 selon que l'élément est dans la solution ou pas.
\end{solution}

Le fichier \textbf{Datas.txt} contient un ensemble de données correspondant à 5 magasins fictifs : t4, t7, t8, t10 et t15. Chaque ligne du fichier contient, le nom du magasin, une liste d'objets et le poids maximal autorisé dans le sac du voleur.

\question A l'aide de la fonction \textbf{perf\_counter} du module \textbf{time}, calculer la durée d'exécution sur les ensembles de données fournis dans le fichier Datas.txt. Les durées seront obtenues en réalisant une moyenne sur 100 appels de la fonction  \textbf{gloutonP}. Noter vos résultats dans les tableaux en annexe A (avec 4 chiffres significatifs sur le temps). 
\begin{solution}
(voir Annexe A) 
\end{solution}
 \subsection{Critère de Valeur}
  Pour notre deuxième algorithme glouton, le critère retenu est \textbf{"placer d'abord dans le sac, les objets de plus grande valeur"}. Ecrire une fonction \textbf{gloutonV} qui accepte les mêmes paramètres que gloutonP et retourne le triplet \textbf{reponse, valeur, poids} comme dans l'exemple ci-après ; réaliser ensuite les mêmes tests à l'aide du fichier  \textbf{datas.txt} pour compléter les tableaux de l'annexe A.
  \vspace{3mm}

\begin{tcolorbox}[enhanced,attach boxed title to top center={yshift=-3mm,yshifttext=-1mm},
  colback=green!5!white,colframe=green!75!black,colbacktitle=green!25!black,
  title=Console Python, fonttitle=\bfseries,
  boxed title style={size=small,colframe=blue!25!black} ]
\begin{pyconsole}
liste=[(7,13), (4,12), (3,8), (3,10)]
gloutonV(liste,30)
\end{pyconsole}
\end{tcolorbox}

\subsection{Critère de Valeur Pondérée par le Poids}

Pour notre dernier algorithme glouton, le critère retenu est \textbf{"placer d'abord dans le sac, les objets dont le rapport \boldmath\( \frac{valeur}{poids} \) est le plus élevé}. Ecrire une fonction \textbf{gloutonVPP} qui accepte les mêmes paramètres que gloutonP et retourne le triplet \textbf{reponse, valeur, poids} comme dans l'exemple ci-après ; réaliser ensuite les mêmes tests à l'aide du fichier  \textbf{datas.txt} pour compléter les tableaux en annexe A.
\begin{solution} 6. - 7.
(voir Annexe B) Algorithmes identiques à gloutonP.  Seule la clé du tri change : la liste "temporaire"  est triée soit  selon le 1er élément (tup[0])  soit selon le rapport $\frac{tup[0]}{tup[1]}$ et dans l'ordre décroissant (reverse=True). 
\end{solution}
  \vspace{5mm}
\begin{tcolorbox}[enhanced,attach boxed title to top center={yshift=-3mm,yshifttext=-1mm},
  colback=green!5!white,colframe=green!75!black,colbacktitle=green!25!black,
  title=Console Python, fonttitle=\bfseries,
  boxed title style={size=small,colframe=blue!25!black} ]
\begin{pyconsole}
liste=[(7,13), (4,12), (3,8), (3,10)]
print(gloutonVPP(liste,30))
\end{pyconsole}
\end{tcolorbox}
\subsection{Comparaison des trois critères}
\question Comparer les durées d'exécution des trois algorithmes gloutons. Commenter. 
\begin{solution} 
(voir Annexe A) gloutonP, gloutonV et gloutonVPP possèdent des durées d'exécution comparables (de l'ordre de $1\times10^{-5}s$) et ceci quel que soit n (pour notre petit échantillon). Les algorithmes gloutons semblent posséder une certaine "stabilité en temps".
\end{solution} 
\question Discuter du choix du critère de sélection des algorithmes gloutons en fonction de $n$.
\begin{solution} 
(voir Annexe A) gloutonP semble le plus mauvais des trois car il ne fait jamais mieux que les deux autres ; gloutonV semble le meilleur pour des petits échantillons mais à partir de n=10 gloutonVPP domine les deux autres. Il faudrait toutefois pouvoir réaliser des tests sur des valeurs de n bien supérieures pour "solidifier" notre hypothèse.
\end{solution} 









\end{questions}
\pagebreak
\appendix
\section{Performances des différentes algorithmes}
\begin{table}[h]
%{\rowcolors{3}{gray!35!black!10}{green!70!yellow!40}
\begin{tabular}{ |p{2.6cm}|p{1.3cm}|p{1.3cm}|p{1.3cm}|p{1.3cm}|p{1.3cm}| }
\hline
\multicolumn{6}{|c|}{Durée d'exécution ( en s)$^*$ de différents algorithme en fonction du nombre d'objets pour le KP} \\
\hline
Algorithme& n=4&n=7&n=8& n=10&n=15\\
\hline
GLOUTONP & 8.490e-6 &1.284e-5&1.967e-5 & 1.857e-5 &2.559e-5 \\
GLOUTONV & 1.059e-5   & 8.202e-6&1.487e-5 & 1.209e-5 &3.243e-5 \\
GLOUTONVPP &2.071e-5 & 7.543e-6 &1.990e-5 & 1.431e-5 &2.716e-5\\
\rowcolor{red}Algorithme Naif    &1.291e-4& 2.129e-3&4.348e-3 & 2.100e-2 &1.137 \\
\hline
\end{tabular}
%}
\caption{Durée d'exécution}
\end{table}
 \vspace{2mm}
  * Calculs effectués sur 1000 itérations pour chaque fonction ; MacBook Pro IntelCore i5 2.4 GHz double coeur\\
  \vspace{5mm}
\begin{table}[h]
%{\rowcolors{3}{gray!35!black!10}{green!70!yellow!40}
\begin{tabular}{ |p{2.6cm}|p{1.3cm}|p{1.3cm}|p{1.3cm}|p{1.3cm}|p{1.3cm}| }
\hline
\multicolumn{6}{|c|}{Réponses$^{**}$ des différents algorithmes en fonction du nombre d'objets pour le KP} \\
\hline
Algorithme& n=4&n=7&n=8& n=10&n=15\\
\hline
GLOUTONP & \tiny\cellcolor{green}([1,1,0,0],\newline 11,25) &\tiny\cellcolor{green}([0, 1, 0, 1, 0, 0, 1], \newline 1735, 169)&\tiny\cellcolor{red}([1, 0, 0, 0, 1, 0, 1, 0], \newline 65, 102) &\tiny\cellcolor{red} ([0, 0, 0, 0, 0, 0, 1, 0, 1, 0], \newline154, 152)&\tiny\cellcolor{red}([1, 1, 0, 0, 0, 0, 1, 0, 0, 0, 1, 1, 0, 0, 1], \newline 1315, 682) \\
GLOUTONV &\tiny\cellcolor{green}([1,1,0,0], \newline11, 25) &\tiny\cellcolor{green}([0, 1, 0, 1, 0, 0, 1], \newline 1735, 169)&\tiny\cellcolor{green}([1, 1, 1, 1, 0, 1, 0, 0],\newline 280, 102) & \tiny\cellcolor{orange}([1, 0, 0, 1, 0, 0, 0, 0, 1, 0],\newline 247, 156) &\tiny\cellcolor{red}([1, 1, 0, 0, 0, 0, 1, 0, 0, 0, 1, 1, 0, 0, 1],  \newline 1315, 682) \\
GLOUTONVPP &\tiny\cellcolor{red}([1, 0, 1, 0],\newline 10,21) & \tiny\cellcolor{red}([1, 1, 1, 0, 0, 0, 0], \newline1478, 140) &\tiny\cellcolor{orange}([1, 1, 1, 1, 0, 0, 0, 1], \newline266, 73) &\tiny \cellcolor{green}([1, 1, 1, 1, 0, 1, 0, 0, 0, 0],\newline  309, 165) &\tiny \cellcolor{green}([0, 1, 1, 0, 1, 1, 1, 0, 1, 0, 0, 0, 1, 1, 0],\newline 1441, 740)\\
Sol. Optimale  &\tiny([1,1,0,0], \newline11, 25)&\tiny ([0, 1, 0, 1, 0, 0, 1], \newline1735, 169)&\tiny([1, 1, 1, 1, 0, 1, 0, 0],\newline 280, 102) & \tiny([1, 1, 1, 1, 0, 1, 0, 0, 0, 0], \newline309, 165) &\tiny([0, 1, 1, 0, 1, 0, 1, 1, 1, 0, 0, 0, 1, 1, 0] ,\newline1458, 749) \\
\hline
\end{tabular}
%}
\caption{Comparaison des réponses obtenues}
\end{table}
 \vspace{2mm}

** Code couleur pour les algorithmes gloutons\\
{\color{green} Réponse optimale ou réponse approchée la plus proche}\\
{\color{orange} Réponse approchée "intermédiaire"}\\
{\color{red} Réponse approchée "éloignée"}
\pagebreak
\section{Question 4,6,7 - Proposition d'algorithmes}
\begin{listing}[h]
\begin{minted}
[frame=leftline,
framesep=2mm,
baselinestretch=1.2,
bgcolor=,
fontsize=\footnotesize,
linenos]{python}
def gloutonP(l,maxi):
    tmp=[x for x in l]
    tmp.sort(key=lambda tup: tup[1],reverse=True)
    valeur=0
    poids=0
    i=0
    solution=[]   
    while i<len(tmp):
        if (poids+tmp[i][1]) <=maxi:
            solution.append(tmp[i])
            poids +=tmp[i][1]
            valeur+=tmp[i][0]
        i+=1
    reponse=[1 if x in solution else 0 for x in l]               
    return reponse,valeur,poids

 \end{minted} 
\end{listing}
\begin{listing}[h]
\begin{minted}
[frame=leftline,
framesep=2mm,
baselinestretch=1.2,
bgcolor=,
fontsize=\footnotesize,
linenos]{python}

def gloutonV(l,maxi=30):
    tmp=[x for x in l]
    l.sort(key=lambda tup: tup[0],reverse=True)
    valeur=0
    poids=0
    solution=[]
    i=0
    while i<len(tmp):
        if (poids+tmp[i][1]) <=maxi:
            solution.append(tmp[i])
            poids +=tmp[i][1]
            valeur+=tmp[i][0]
        i+=1
    reponse=[1 if x in solution else 0 for x in tmp]       
    return reponse,valeur,poids
 \end{minted} 
\end{listing}

\begin{listing}[h]
\begin{minted}
[frame=leftline,
framesep=2mm,
baselinestretch=1.2,
bgcolor=,
fontsize=\footnotesize,
linenos]{python}

def gloutonVPP(l,maxi=30):
    tmp=[x for x in l]
    l.sort(key=lambda tup: tup[0]/tup[1],reverse=True)
    valeur=0
    poids=0
    solution=[]
    i=0
    while i<len(tmp):
        if (poids+tmp[i][1]) <=maxi:
            solution.append(tmp[i])
            poids +=tmp[i][1]
            valeur+=tmp[i][0]
        i+=1
    reponse=[1 if x in solution else 0 for x in tmp]        
    return reponse,valeur,poids
 \end{minted} 
\end{listing}
\pagebreak

\end{document}

