% !TeX document-id = {8613a224-84c8-4428-9607-d4a0352535f7}
%pdflatex -shell-escape doc4.tex
%pythontex doc4.tex
%pdflatex -shell-escape doc4.tex
% !TeX TXS-program:compile = txs:///pdflatex/[--shell-escape]
\documentclass[a4paper,12,%answers
]{exam}
\usepackage{fourier}
\usepackage{geometry}
\usepackage{array,multicol,multirow,enumerate,eurosym,latexsym,fourier,bbding,pifont}

\geometry{vmargin=10mm,hmargin=5mm}
\usepackage{fourier}
\usepackage[utf8]{inputenc}
\usepackage{geometry}
\geometry{vmargin=10mm,hmargin=5mm}
%\usepackage[latin1]{inputenc}
\usepackage[T1]{fontenc}
\usepackage[francais]{babel}
\usepackage{listings}
\usepackage{pythontex}
\usepackage{minted}
%renewcommand\listingscaption{Code Source}
\usepackage[table]{xcolor}
\usepackage{graphicx}
\graphicspath{ {/Users/the_brat/Documents/Mon_Latex} }
\usepackage{tabto}
%\usepackage{minibox}
\definecolor{ivory}{rgb}{1.0, 1.0, 0.94}
\definecolor{cadetgrey}{rgb}{0.57, 0.64, 0.69}
\usepackage{tikz,lipsum,lmodern}
\usepackage[most]{tcolorbox}

\usepackage[titletoc]{appendix}
\usepackage{wrapfig}
%\usepackage[many]{tcolorbox}%https://tex.stackexchange.com/questions/262052/lstlisting-framed-with-the-caption-inside-the-frame
%\tcbuselibrary{listings}


\newcommand{\expandpyconc}[1]{\expandafter\reallyexpandpyconc\expandafter{#1}}
\newcommand{\reallyexpandpyconc}[1]{\pyconc{exec(compile(open('#1', 'rb').read(), '#1', 'exec'))}}

   

\begin{document}
\lhead{Lycée Jean Monnet - \textit{NSI}}
\chead{}
\rhead{\textit{Année} 2019/2020}

\lfoot{                      }\cfoot{Page \thepage}\rfoot{\textsf{Aude Duhem et Patrice Nicolas}}
NOM : .................. \hfill PRENOM : ............................ \hfill Classe : .............\\
\hrule

\begin{center}
	\textbf{\Large{Evaluation - Les algorithmes gloutons }}\end{center}
\hrule
\vskip0.5cm
	\begin{questions}
	\question On suppose dans cette question, que l'on dispose des pièces de monnaie suivantes (sans limite d'effectif) : 5 \euro,\,2 \euro, 1 \euro,\,50 centimes, \,20 centimes,\,10 centimes,\,5 centimes, \,2 centimes et 1 centime.\\
	Pour 11,67 \euro\, payé avec un billet de 20 \euro, le rendu de monnaie en pièces obtenu à l'aide d'un algorithme glouton est : \\
	Cocher la bonne réponse:
			\begin{checkboxes}
			\choice Réponse A :\\
			2 pièces de 5 \euro, \,1 pièce de 1 \euro,\,1 pièce de 50 centimes, 1 pièce de 10 centimes, 1 pièce de 5 centimes et 1 pièce de 2 centimes.
			\choice Réponse B :\\
			1 pièce de 5 \euro, \,3 pièces de 1 \euro,\,1 pièce de 20 centimes, 1 pièce de 10 centimes, 1 pièce de 2 centimes et 1 pièce de 1 centime.
			\choice Réponse C : \\
			1 pièce de 5 \euro, \,1 pièce de 1 \euro,\,2 pièce de 2 \euro,\,3 pièces de 10 centimes, 1 pièce de 2 centimes et 1 pièce de 1 centime.
			\choice Réponse D :\\ 1 pièce de 5 \euro, \,1 pièce de 1 \euro,\,2 pièce de 2 \euro,\,1 pièce de 20 centimes, 1 pièce de 10 centimes,  1 pièce de 2 centimes et 1 pièce de 1 centime.
		\end{checkboxes}	

\begin{solution}
C'est la réponse D qui  convient.

\end{solution}
\medskip
	\question On dispose d'un système de monnaie (sans limite d'effectif) de 4$\star$, 3$\star$ et 1$\star$, on cherche la façon de payer 6 $\star$ selon le principe glouton. La répartion est : \\
Cocher la bonne réponse:
\begin{checkboxes}
	\choice Réponse A : 3 $\star$ , 3 $\star$
	\choice Réponse B : 3 $\star$ , 1 $\star$, 1 $\star$, 1 $\star$
	\choice Réponse C : 4 $\star$ , 1 $\star$, 1 $\star$
		\choice Réponse D : 1 $\star$, 1 $\star$, 1 $\star$, 1 $\star$,1 $\star$, 1 $\star$
\end{checkboxes}		

\begin{solution}
	C'est la réponse C qui  convient.
	
\end{solution}
\medskip


\question L'algorithme glouton ( Cocher la bonne réponse) :

\begin{checkboxes}
\choice Réponse A :  assure de trouver une solution optimale
\choice Réponse B : entraîne un coût en temps exponentiel
\choice Réponse B : donne parfois une solution optimale
\end{checkboxes}
\begin{solution}
	C
\end{solution}
\medskip
\question Soit une liste L=[(7,13),(4,12),(3,8),(3,10)] ; en python, l'instruction \textbf{L.sort(key = lambda a : a[1], reverse=True) }:\\
Cocher la bonne réponse:
\begin{checkboxes}
\choice Réponse A :  trie  les tuples par ordre décroissant en fonction de l'élément d'indice1
\choice Réponse B :  trie à partir premier élément de la liste
\choice Réponse C :  trie  les tuples par ordre croissant en fonction de l'élément d'indice 0
\end{checkboxes}
\begin{solution}
	A
\end{solution}
\medskip
\question Quels mots complètent (dans l'ordre) la définition du problème du sac à dos : \\
\medskip
\textbf{«Étant  donné  plusieurs  objets  possédant chacun  un  ....................  et  une ..............  et  étant  donné  un  poids ..............  pour  le  sac,  quels  objets faut-il  mettre  dans  le  sac  de  manière  à  .....................  la  valeur  totale  sans  dépasser  le  poids maximal autorisé pour le sac?»}\\
\medskip
Cocher la bonne réponse:

\begin{checkboxes}
	\choice Réponse A : indice, liste, minimal, minimiser
\choice Réponse B :  poids, valeur, minimal, maximiser
\choice Réponse C :  poids, valeur, maximal, maximiser
\end{checkboxes}
\begin{solution}
	C
\end{solution}
\medskip
\question Soit une liste de tuples (valeur,poids) L=[(15, 2),(100, 20),(90, 20),(60, 30),(40, 40),(15, 30),(10, 60),(1, 1)] . On dispose d'un sac de poids maximal autorisé $p_{max}$ = 90 et on utilise l'algorithme glouton avec un critère sur la valeur. L'algorithme propose une solution avec un sac de valeur totale :\\
Cocher la bonne réponse:
\begin{checkboxes}
	\choice Réponse A : 245\\
\choice Réponse B : 265\\
\choice Réponse C : 266
\end{checkboxes}
\begin{solution}
	C
\end{solution}

\end{questions}
\end{document}    


