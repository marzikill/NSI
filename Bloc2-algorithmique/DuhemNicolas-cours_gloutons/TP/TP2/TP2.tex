
% !TeX document-id = {117ac6b5-8dd4-4613-9231-0768b5815107}
%pdflatex -shell-escape TP2.tex
%pythontex TP2.tex
%pdflatex -shell-escape tp.tex
% !TeX TXS-program:compile = txs:///pdflatex/[--shell-escape]

\documentclass[12pt,french]{article}
\usepackage[utf8]{inputenc}
\usepackage{array,multicol,multirow,enumerate,eurosym,latexsym,fourier,bbding,pifont}
\usepackage{fourier}
\usepackage{graphicx,pst-all}
\usepackage{tabularx}
\usepackage [alwaysadjust]{paralist}
\usepackage{amsmath,amsfonts,amsthm,amssymb,geometry}
\usepackage{fancyhdr}
\usepackage{mathrsfs}  
\usepackage{pstricks,pst-plot,pst-text,pst-tree,pstricks-add,pst-eps,pst-fill,pst-node,pst-math}
\usepackage{euscript,amsfonts,eepic,color}
\usepackage{ifthen,fp}
\newcommand{\Calig}[1]{\ensuremath{\mathscr{#1}}}              
\usepackage{babel}
\usepackage{xcolor}
\usepackage{minted}
\usepackage{pythontex}
\usepackage{multicol}
\usepackage[most]{tcolorbox}
\usepackage{fancyhdr}
\setlength{\parindent}{0pt}
\usepackage{ulem}
\usepackage[np]{numprint}
\geometry{vmargin=15mm,hmargin=5mm}
\pagestyle{empty}
\setlength\columnsep{5mm}
\renewcommand{\thesection}{\Roman{section}}
\renewcommand{\thesubsection}{\Alph{subsection}}
\renewcommand{\thesubsubsection}{\arabic{subsubsection}}
\newcounter{npb}
\setcounter{npb}{0}
\newcommand{\exo}{
    \stepcounter{npb}
    {\textbf{$\triangleright$ \underline{Exercice \arabic{npb} }}}
}
\newcounter{sf}
\setcounter{sf}{0}
\newcommand{\s}{
    \stepcounter{sf}
    {\textbf{ \fbox{SF \arabic{sf} }}}
}
\usepackage{lscape}
\usepackage{tikz}
\usepackage{metalogo}
\usepackage{hyperref}
\newcommand{\expandpyconc}[1]{\expandafter\reallyexpandpyconc\expandafter{#1}}
\newcommand{\reallyexpandpyconc}[1]{\pyconc{exec(compile(open('#1', 'rb').read(), '#1', 'exec'))}}

\newenvironment{pyconcodeblck}[1]%
{\newcommand{\snippetfile}{snippet-#1.py}
	\VerbatimEnvironment
	\begin{VerbatimOut}{\snippetfile}}%
	{\end{VerbatimOut}%
	\expandpyconc{\snippetfile}}
\newcommand{\typesetcode}[1]{\inputpygments{python}{snippet-#1.py}}


\begin{pyconcodeblck}{temp19}
def extraire(t,n):
	for i in range(1,len(t)):
		v = t[i][n]
		j = i-1
		while(j>=0 and t[j][n]>v):
			t[j+1],t[j] = t[j],t[j+1]
			j = j-1
		t[j+1][n] = v
	return t
\end{pyconcodeblck}
\begin{pyconcodeblck}{temp14}
def gloutonP(l,maxi):
	tmp=[x for x in l]
	tmp.sort(key=lambda tup: tup[1],reverse=True)
	valeur=0
	poids=0
	i=0
	solution=[]   
	while i<len(tmp):
		if (poids+tmp[i][1]) <=maxi:
			solution.append(tmp[i])
			poids +=tmp[i][1]
			valeur+=tmp[i][0]
		i+=1
	reponse=[1 if x in solution else 0 for x in l]        
	
	return reponse,valeur,poids
\end{pyconcodeblck}

\begin{pyconcodeblck}{temp15}
def gloutonV(l,maxi=30):
	tmp=[x for x in l]
	l.sort(key=lambda tup: tup[0],reverse=True)
	valeur=0
	poids=0
	i=0
	solution=[]
	while i<len(tmp):
		if (poids+tmp[i][1]) <=maxi:
			solution.append(tmp[i])
			poids +=tmp[i][1]
			valeur+=tmp[i][0]
		i+=1
	reponse=[1 if x in solution else 0 for x in tmp]
	
	return reponse,valeur,poids
\end{pyconcodeblck}

\begin{pyconcodeblck}{temp16}
def gloutonVPP(l,maxi=30):
	tmp=[x for x in l]
	l.sort(key=lambda tup: tup[0]/tup[1],reverse=True)
	valeur=0
	poids=0
	solution=[]
	i=0
	while i<len(tmp):
		if (poids+tmp[i][1]) <=maxi:
			solution.append(tmp[i])
			poids +=tmp[i][1]
			valeur+=tmp[i][0]
		i+=1
	reponse=[1 if x in solution else 0 for x in tmp]
	
	return reponse,valeur,poids
	
\end{pyconcodeblck}
\begin{document}

  \lhead{Lycée Jean Monnet - \textit{NSI}}
    \chead{}
    \rhead{\textit{Année} 2019/2020}
      \renewcommand{\headrulewidth}{0.5pt}
      \lfoot{                      }\cfoot{Page \thepage}\rfoot{\textsf{Aude Duhem et Patrice Nicolas}}
    \pagestyle{fancy}
    \renewcommand{\footrulewidth}{0.4pt}
\begin{center}
\textbf{\Large{TP2 - Le problème du rendu de monnaie}}\end{center}
\hrule
\medskip
\section{Retour sur les Tris}
\begin{enumerate}
	
	\item Ecrire une fonction \textbf{extraire} qui accepte en paramètres, une liste de liste  \textbf{l} et un entier \textbf{n}  dans [0; 2] .La fonction doit retourner la liste triée en fonction du n-ième  élément de chaque sous-liste. On pourra choisir le tri insertion par exemple.
	\begin{tcolorbox}[enhanced,attach boxed title to top center={yshift=-3mm,yshifttext=-1mm},
		colback=green!5!white,colframe=green!75!black,colbacktitle=green!25!black,
		title=Console Python, fonttitle=\bfseries,
		boxed title style={size=small,colframe=blue!25!black} ]
		\begin{pyconsole}
L= [ [1, 5, 6], [8, 10, 2], [3, 3, 5],[4, 8, 1] ]
extraire(L,1)
extraire(L,2)
		\end{pyconsole}
	\end{tcolorbox}
	
	
	
	\item Pourquoi le code précédent ne fonctionne-t-il pas (tel quel) avec une liste de tuples?

	\item A l'aide de l'exercice 7, expliquer le sens de l'instruction \mintinline{python}{List.sort(key = lambda a : a[1]) }
\end{enumerate}	

\section{Algorithme du sac à dos}
	\subsection{Critère de Poids}
	
	Nous allons écrire un premier algorithme glouton de critère \textbf{"placer d'abord dans le sac, les objets les plus lourds."} En cas d'égalité, on prendra "au hasard" un objet parmi les indécis. On pourra utiliser la méthode \textbf{list.sort()} comme définie sur la documentation officielle \href{https://docs.python.org/fr/3/howto/sorting.html}{Python}	
	\begin{enumerate}
	\item Ecrire une fonction \textbf{gloutonP} de paramètres \textbf{l} et \textbf{maxi}, respectivement une liste de tuples (valeur en €, poids en kg) et un entier correspondant au poids maximal supporté par le sac.  La fonction doit retourner un triplet \textbf{reponse, valeur, poids} comme dans l'exemple ci-dessous:
	\begin{tcolorbox}[enhanced,attach boxed title to top center={yshift=-3mm,yshifttext=-1mm},
		colback=green!5!white,colframe=green!75!black,colbacktitle=green!25!black,
		title=Console Python, fonttitle=\bfseries,
		boxed title style={size=small,colframe=blue!25!black} ]
\begin{pyconsole}
liste=[(7,13), (4,12), (3,8), (3,10)]
gloutonP(liste,30)
		\end{pyconsole}
	\end{tcolorbox}
	Le fichier \textbf{Datas.txt} contient un ensemble de données correspondant à 5 magasins fictifs : t4, t7, t8, t10 et t15. Chaque ligne du fichier contient, le nom du magasin, une liste d'objets et le poids maximal autorisé dans le sac du voleur.	
	\item A l'aide de la fonction \textbf{perf\_counter} du module \textbf{time}, calculer la durée d'exécution sur les ensembles de données fournis dans le fichier Datas.txt. Les durées seront obtenues en réalisant une moyenne sur 100 appels de la fonction \textbf{gloutonP}. Noter vos résultats dans les tableaux en annexe A (avec 4 chiffres significatifs sur le temps). 
\end{enumerate}
	\subsection{Critère de Valeur}
	
	Pour notre deuxième algorithme glouton, le critère retenu est \textbf{"placer d'abord dans le sac, les objets de plus grande valeur"}. Ecrire une fonction \textbf{gloutonV} qui accepte les mêmes paramètres que gloutonP et qui retourne le triplet \textbf{reponse, valeur, poids} comme dans l'exemple ci-après ; réaliser ensuite les mêmes tests à l'aide du fichier  \textbf{datas.txt} pour compléter les tableaux de l'annexe A.
	\begin{tcolorbox}[enhanced,attach boxed title to top center={yshift=-3mm,yshifttext=-1mm},
		colback=green!5!white,colframe=green!75!black,colbacktitle=green!25!black,
		title=Console Python, fonttitle=\bfseries,
		boxed title style={size=small,colframe=blue!25!black} ]
\begin{pyconsole}
liste=[(7,13), (4,12), (3,8), (3,10)]
gloutonV(liste,30)
		\end{pyconsole}
	\end{tcolorbox}
	
	
	\subsection{Critère de Valeur Pondérée par le Poids}
	
	Pour notre dernier algorithme glouton, le critère retenu est \textbf{"placer d'abord dans le sac, les objets dont le rapport \boldmath\( \frac{\text{valeur}}{\text{poids}} \) est le plus élevé}.  Ecrire une fonction \textbf{gloutonVPP} qui accepte les mêmes paramètres que gloutonP et retourne le triplet \textbf{reponse, valeur, poids} comme dans l'exemple ci-après ; réaliser ensuite les mêmes tests à l'aide du fichier  \textbf{datas.txt} pour compléter les tableaux en annexe A.
	\begin{tcolorbox}[enhanced,attach boxed title to top center={yshift=-3mm,yshifttext=-1mm},
		colback=green!5!white,colframe=green!75!black,colbacktitle=green!25!black,
		title=Console Python, fonttitle=\bfseries,
		boxed title style={size=small,colframe=blue!25!black} ]
\begin{pyconsole}
liste=[(7,13), (4,12), (3,8), (3,10)]
gloutonVPP(liste,30)
\end{pyconsole}
	\end{tcolorbox}
	\subsection{Comparaison des trois critères}
	\begin{enumerate}
\item  Comparer les durées d'exécution des trois algorithmes gloutons. Commenter. 	
	\item Discuter du choix du critère de sélection des algorithmes gloutons en fonction de $n$.
	\end{enumerate}
\hrule
\small	
	\begin{center}
		\textbf{Annexe A : comparaison des performances des différents algorithmes}
	\end{center}


	%{\rowcolors{3}{gray!35!black!10}{green!70!yellow!40}
	\begin{tabular}{ |p{2.6cm}|p{2.6cm}|p{2.6cm}|p{2.6cm}|p{2.6cm}|p{2.6cm}| }
		\hline
		\multicolumn{6}{|c|}{Durée d'exécution ( en s)$^*$ de différents algorithme en fonction du nombre d'objets pour le KP} \\
		\hline
		Algorithme& n=4&n=7&n=8& n=10&n=15\\
		\hline
		GLOUTONP &&&&& \\
		&&&&&\\
		\hline
		GLOUTONV &&&&& \\
		&&&&&\\
		\hline
		GLOUTONVPP &&&&&\\
		&&&&&\\
		\hline
	Algorithme Naif    &1.291e-4& 2.129e-3&4.348e-3 & 2.100e-2 &1.137 \\
		\hline
	\end{tabular}
	%}
	

\vspace{2mm}
* Calculs effectués sur 1000 itérations pour chaque fonction ; MacBook Pro IntelCore i5 2.4 GHz double coeur
\begin{table}[h]
	%{\rowcolors{3}{gray!35!black!10}{green!70!yellow!40}
	\begin{tabular}{ |p{2.5cm}|p{1.5cm}|p{2.8cm}|p{3.2cm}|p{3.6cm}|p{4,5cm}| }
		\hline
		\multicolumn{6}{|c|}{Réponses$^{**}$ des différents algorithmes en fonction du nombre d'objets pour le KP} \\
		\hline
		Algorithme& n=4&n=7&n=8& n=10&n=15\\
		\hline
		GLOUTONP &&&&& \\
		&&&&& \\
		\hline
		GLOUTONV &&&&& \\
			&&&&& \\
		\hline
		GLOUTONVPP &&&&&\\
			&&&&& \\
		\hline
		Sol. Optimale  &\small([1,1,0,0],\newline11, 25)&\small ([0, 1, 0, 1, 0, 0, 1], \newline1735, 169)&\small([1, 1, 1, 1, 0, 1, 0, 0],\newline 280, 102) & \small([1, 1, 1, 1, 0, 1, 0, 0, 0, 0], 309, 165) &\small([0, 1, 1, 0, 1, 0, 1, 1, 1, 0, 0, 0, 1, 1, 0] ,1458, 749) \\
		\hline
	\end{tabular}
	%}
	
\end{table}
 
\normalsize
\hrule
\medskip
\begin{center} 
\textbf{Pour aller plus loin ... : exercice 8 de la feuille d'exercices} \end{center}



\end{document}








